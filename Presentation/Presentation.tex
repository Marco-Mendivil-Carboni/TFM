\documentclass{beamer}

\usepackage[T1]{fontenc}
\usepackage[utf8]{inputenc}

\usepackage{lmodern}

\usepackage[spanish]{babel}

\usepackage{mathtools,amssymb}

\usepackage{graphicx}
\usepackage{xcolor}
\usepackage{tikz}

\usepackage{booktabs}

\usepackage{pdfpc}

\usetheme{CambridgeUS}

\usecolortheme{seahorse}

\definecolor{color1}{HTML}{d81e2c}
\definecolor{color2}{HTML}{a31cc5}
\definecolor{color3}{HTML}{194bb2}
\definecolor{color4}{HTML}{169f62}

\usetikzlibrary{calc}

\pdfpcsetup{duration=30}

\AtBeginSection[]
{
    \begin{frame}{Índice}
        \tableofcontents[currentsection]
    \end{frame}
}

\setbeamertemplate{sections/subsections in toc}[square]
\setbeamertemplate{items}[square]

\title[Trabajo fin de Máster]{Modelo polimérico de la cromatina: Estudio \\de su organización dentro del núcleo celular}
\author[]
{
    \texorpdfstring{{\large Mendívil Carboni, Marco}\\\vspace{1ex}
    Falo Forniés, Fernando\inst{1} (dir.) \and Sáinz-Agost, Alejandro\inst{1} (dir.)}{Marco Mendívil Carboni}
}
\institute[Universidad de Zaragoza]
{
    \inst{1}
    Universidad de Zaragoza, Facultad de Ciencias\\
    Departamento de Física de la Materia Condensada 
}

\begin{document}

\frame{\titlepage}

\begin{frame}{Índice}
    \tableofcontents
\end{frame}

\section{Introducción}

\subsection{Motivación y objetivos}

\begin{frame}
    \begin{block}{Motivación}
        \begin{itemize}
            \item La cromatina, el complejo de \alert{ADN y proteínas} que constituye el material génetico del núcleo celular, es un ejemplo paradigmático de \alert{sistema complejo}.
            \item Este sistema tiene un \alert{interés biológico} inmenso.
        \end{itemize}
    \end{block}
    \begin{block}{Objetivos}
        Desarrollo de un \alert{modelo} de la cromatina que nos permita, mediante simulaciones numéricas, reproducir las \alert{distribuciones espaciales} de la misma que se observan experimentalmente. Para ello:
        \begin{itemize}
            \item Estudiaremos las propiedades más importantes de la cromatina y diseñaremos un modelo de la misma.
            \item Detallaremos los aspectos más importantes de la metodología.
            \item Mostraremos finalmente los resultados de las simulaciones.
            \item Valoraremos si hemos cumplido con el objetivo.
        \end{itemize}
    \end{block}
\end{frame}

\subsection{La física de polímeros}

\begin{frame}{Modelos ideales}
    \begin{figure}
        \centering
        \begin{tikzpicture}[scale=1.5]
            \draw[black,semithick] (+0.0,+0.0) -- (+0.9,+0.1);
            \draw[black,semithick] (+0.9,+0.1) -- (+1.2,+0.9);
            \draw[black,semithick] (+1.2,+0.9) -- (+1.9,+1.2);
            \draw[black,semithick] (+1.9,+1.2) -- (+2.6,+1.2);
            \draw[black,semithick] (+2.6,+1.2) -- (+3.2,+0.6);
            \draw[black,semithick] (+3.2,+0.6) -- (+3.4,+0.0);
            \draw[black,semithick] (+3.4,+0.0) -- (+2.8,-0.4);
            \draw[black,semithick] (+2.8,-0.4) -- (+1.9,-0.5);
            \draw[black,semithick] (+1.9,-0.5) -- (+1.6,+0.3);
            \draw[black,semithick] (+1.6,+0.3) -- (+0.9,+0.7);
            \draw[black,semithick] (+0.9,+0.7) -- (+0.3,+1.4);
            \draw[black,semithick] (+0.3,+1.4) -- (-0.1,+0.8);
            \draw[black,semithick] (-0.1,+0.8) -- (-0.9,+0.6);
            \draw[-latex,color1,semithick] (+3.8,+1.6) -- (+2.6,+1.2) node[midway,above]{$\bar{r}_i$};
            \draw[-latex,color1,semithick] (+3.8,+1.6) -- (+4.4,+1.0) node[midway,left]{$\bar{b}_i$};
            \draw[latex-latex,color1,semithick,dashed] (+2.6,+1.2) -- (+3.2,+0.6) node[midway,left]{$l_i$};
            \draw[color1,semithick,dashed] (+2.6,+1.2) -- (+3.3,+1.2);
            \draw[color1,semithick] (+3.1,+1.2) arc (0:-45:0.5) node[midway,right]{$\theta_i$};
            \draw[latex-latex,color1,semithick,dashed] (+2.6,+1.2) -- (+0.3,+1.4) node[midway,above]{$d_{ij}$};
            \draw[black,semithick,fill] (+3.8,+1.6) circle (1pt) node[anchor=south]{$O$};
        \end{tikzpicture}
    \end{figure}
    \begin{block}{Distancia (espacial) promedio}
        \begin{equation}
            d(s)\propto s^{1/2}
        \end{equation}
    \end{block}
    \begin{block}{Probabilidad de contacto}
        \begin{equation}
            P(s)\propto s^{-3/2}
        \end{equation}
    \end{block}
\end{frame}

\begin{frame}
    \begin{block}{Elasticidad}
        \begin{equation}
            V_e=\sum_{i=1}^{N-1}\frac{1}{2}k_e(l_i-l_0)^2
        \end{equation}
    \end{block}
    \begin{block}{Flexibilidad}
        \begin{equation}
            V_b=\sum_{i=1}^{N-2} k_b (1-\cos(\theta_i))
        \end{equation}
    \end{block}
    \begin{block}{Longitud de persistencia}
        \begin{equation}
            l_P=-\frac{l_0}{\ln(\coth(\beta k_b)-1/(\beta k_b))}\approx\beta k_bl_0
        \end{equation}
    \end{block}
    Para $s\ll l_P/l_0$ $d(s)\propto s^{1}$ y $P(s)\propto s^{-3}$.
\end{frame}

\begin{frame}{Modelos reales}
    \begin{block}{Potencial de Wang-Frenkel}
        \begin{equation}
            V_{WF}=\sum_{i=0}^{N-1}\sum_{j=i+1}^{N-1}\varepsilon\left(\left(\frac{\sigma}{d_{ij}}\right)^{2}-1\right)\left(\left(\frac{2\sigma}{d_{ij}}\right)^{2}-1\right)^{2}
        \end{equation}
    \end{block}
    \begin{block}{Interacción puramente repulsivas}
        \begin{equation}
            d(s)\propto s^{\nu}  \quad \text{y} \quad P(s)\propto s^{-\nu(3+\theta)}
        \end{equation}
        donde $\nu\simeq0.588$ y $\theta\simeq0.71$ luego $\nu(3+\theta)\simeq2.2$.
    \end{block}
    \begin{block}{Interacción atractiva}
        \begin{equation}
            d(s)\propto s^{1/3}  \quad \text{y} \quad P(s)\propto s^{-1}.
        \end{equation}
    \end{block}
\end{frame}

\section{La cromatina}

\subsection{Propiedades biofísicas}

\begin{frame}
    La unión del ADN con las proteínas se forma con el principal objetivo de \alert{concentrar} el material genético para que pueda caber dentro del núcleo celular.
    \begin{figure}
        \centering
        \includegraphics[width=0.5\textwidth]{../Multimedia/Images/Chromatin-Structure.png}
    \end{figure}
\end{frame}

\begin{frame}
    \begin{figure}
        \centering
        \includegraphics[width=0.5\textwidth]{../Multimedia/Images/Hi-C-example.png}
    \end{figure}
\end{frame}

\subsection{Modelo polimérico}

\begin{frame}
    \begin{figure}
        \centering
        \includegraphics[scale=0.75]{../Plots/sequence.pdf}
    \end{figure}
    \begin{enumerate}
        \item LADh (A): Dominios Asociados a la Lámina heterocromáticos
        \item LNDe (B): Dominios No asociados a la Lámina eucromáticos
    \end{enumerate}
\end{frame}

\begin{frame}
    \begin{table}
        \centering
        \begin{tabular}{c c c}
            \toprule
            parámetro       & valor (SI)         & valor (UR)          \\
            \midrule
            $l_0$           & $33$ nm            & $1$ $l_u$           \\
            $k_e$           & $0.48$ pN/nm       & $128$ $e_u$/$l_u^2$ \\
            $k_b^A$         & $4.11$ pN$\cdot$nm & $1$ $e_u$           \\
            $\sigma$        & $33$ nm            & $1$ $l_u$           \\
            $\epsilon^{AA}$ & $2.06$ pN$\cdot$nm & $0.5$ $e_u$         \\
            $\epsilon'$     & $32.9$ pN$\cdot$nm & $8$ $e_u$           \\
            \bottomrule
        \end{tabular}
    \end{table}
\end{frame}

\section{Metodología}

\subsection{Dinámica Molecular}

\begin{frame}
    \dots
\end{frame}

\subsection{Programación en GPUs}

\begin{frame}
    \dots
\end{frame}

\subsection{Cálculo del volumen excluido}

\begin{frame}
    \begin{figure}
        \centering
        \begin{tikzpicture}[scale=1.5]
            \draw[color1,fill=color1,fill opacity=0.25] (1.7,1.2) circle (0.5) node[opacity=1] {0};
            \draw[color1,fill=color1,fill opacity=0.25] (2.9,2.1) circle (0.5) node[opacity=1] {1};
            \draw[color1,fill=color1,fill opacity=0.25] (2.1,2.4) circle (0.5) node[opacity=1] {2};
            \draw[color1,fill=color1,fill opacity=0.25] (0.9,2.1) circle (0.5) node[opacity=1] {3};
            \draw[color1,fill=color1,fill opacity=0.25] (2.8,2.8) circle (0.5) node[opacity=1] {4};
            \draw[color1,fill=color1,fill opacity=0.25] (0.6,2.8) circle (0.5) node[opacity=1] {5};
            \foreach \i in {0,...,4} {\draw[black] (\i,0) -- (\i,4); \draw[black] (0,\i) -- (4,\i);}
            \foreach \i in {0,...,3} {\foreach \j [evaluate=\num using int(\i+4*\j)] in {0,...,3} {\node[above right] at (\i,\j) {\num};}}
            \draw[latex-latex,color1,semithick,dashed] (0.5,0.0) -- (0.5,1.0) node[midway,right] {$2\sigma$};
        \end{tikzpicture}
    \end{figure}
\end{frame}

\begin{frame}
    \begin{figure}
        \centering
        \includegraphics[scale=0.75]{../Plots/performance.pdf}
    \end{figure}
\end{frame}

\section{Resultados: organización espacial}

\subsection{Configuración convencional}

\begin{frame}
    \begin{figure}
        \centering
        \includegraphics[width=0.5\textwidth]{../Multimedia/Images/Conventional-QS.png}
    \end{figure}
\end{frame}

\begin{frame}
    \begin{figure}
        \centering
        \includegraphics[scale=0.75]{../Plots/30386-0.2-0.4-0.0-0.0/rcd.pdf}
    \end{figure}
\end{frame}

\begin{frame}
    \begin{figure}
        \centering
        \includegraphics[scale=0.75]{../Plots/30386-0.2-0.4-0.0-0.0/sd.pdf}
    \end{figure}
\end{frame}

\begin{frame}
    \begin{figure}
        \centering
        \includegraphics[scale=0.75]{../Plots/30386-0.2-0.4-0.0-0.0/cp.pdf}
    \end{figure}
\end{frame}

\begin{frame}
    \begin{figure}
        \centering
        \includegraphics[scale=0.70]{../Plots/30386-0.2-0.4-0.0-0.0/cm.pdf}
    \end{figure}
\end{frame}

\subsection{Configuración invertida}

\begin{frame}
    \begin{figure}
        \centering
        \includegraphics[width=0.5\textwidth]{../Multimedia/Images/Inverted-QS.png}
    \end{figure}
\end{frame}

\begin{frame}
    \begin{figure}
        \centering
        \includegraphics[scale=0.75]{../Plots/30386-0.2-0.0-0.0-0.0/rcd.pdf}
    \end{figure}
\end{frame}

\begin{frame}
    \begin{figure}
        \centering
        \includegraphics[scale=0.70]{../Plots/30386-0.2-0.0-0.0-0.0/cm.pdf}
    \end{figure}
\end{frame}

\section{Conclusiones y trabajo futuro}

\begin{frame}
    \begin{figure}
        \centering
        \includegraphics[width=1.00\textwidth]{../Multimedia/Images/decay-length.png}
    \end{figure}
\end{frame}

\begin{frame}
    \begin{figure}
        \centering
        \includegraphics[width=0.75\textwidth]{../Multimedia/Images/bleb.png}
    \end{figure}
\end{frame}

\end{document}
