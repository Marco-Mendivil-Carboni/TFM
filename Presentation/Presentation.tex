\documentclass{beamer}

\usepackage[T1]{fontenc}
\usepackage[utf8]{inputenc}

\usepackage{lmodern}

\usepackage[spanish]{babel}

\usepackage{mathtools,amssymb}

\usepackage{graphicx}
\usepackage{xcolor}
\usepackage{tikz}

\usepackage{booktabs}

\usetheme{CambridgeUS}

\usecolortheme{seahorse}

\definecolor{color1}{HTML}{d81e2c}
\definecolor{color2}{HTML}{a31cc5}
\definecolor{color3}{HTML}{194bb2}
\definecolor{color4}{HTML}{169f62}

\usetikzlibrary{calc}

\AtBeginSection[]
{
    \begin{frame}{Índice}
        \tableofcontents[currentsection]
    \end{frame}
}

\setbeameroption{show notes}

\setbeamertemplate{note page}{\vskip 0.5em \insertnote}
\addtobeamertemplate{note page}{}{\thispdfpagelabel{notes:\insertframenumber}}

\setbeamertemplate{sections/subsections in toc}[square]
\setbeamertemplate{items}[square]

\title[Trabajo fin de Máster]{Modelo polimérico de la cromatina: Estudio \\de su organización dentro del núcleo celular}
\author[]
{
    \texorpdfstring{{\large Mendívil Carboni, Marco}\\\vspace{1ex}
    Falo Forniés, Fernando\inst{1} (dir.) \and Sáinz-Agost, Alejandro\inst{1} (dir.)}{Marco Mendívil Carboni}
}
\institute[Universidad de Zaragoza]
{
    \inst{1}
    Universidad de Zaragoza, Facultad de Ciencias\\
    Departamento de Física de la Materia Condensada 
}

\begin{document}

\frame{\titlepage}

\begin{frame}{Índice}
    \tableofcontents
\end{frame}

\section{Introducción}

\subsection{Motivación y objetivos}

\begin{frame}
    \begin{block}{Motivación}
        \begin{itemize}
            \item La cromatina, el complejo de \alert{ADN y proteínas} que constituye el material génetico del núcleo celular, es un ejemplo paradigmático de \alert{sistema complejo}.
            \item Este sistema tiene un \alert{interés biológico} inmenso.
        \end{itemize}
    \end{block}
    \begin{block}{Objetivos}
        Desarrollo de un \alert{modelo} de la cromatina que nos permita, mediante simulaciones numéricas, reproducir las \alert{distribuciones espaciales} de la misma que se observan experimentalmente. Para ello:
        \begin{itemize}
            \item Estudiaremos las propiedades más importantes de la cromatina y diseñaremos un modelo de la misma.
            \item Detallaremos los aspectos más importantes de la metodología.
            \item Mostraremos finalmente los resultados de las simulaciones.
            \item Valoraremos si hemos cumplido con el objetivo.
        \end{itemize}
    \end{block}
    \note[item]{En el último siglo se ha tendido a diseccionar los sistemas biológicos en sus partes constituyentes más simples para facilitar su entendimiento, de forma similar a como se ha hecho en física. Esta estrategia reduccionista ha sido muy exitosa: ha impulsado el desarrollo de la biología molecular y sus resultados, como el descubrimiento de la estructura del ADN, han sido fundamentales para alcanzar la comprensión actual de los mecanismos de la vida.}
    \note[item]{Pero ahora son cada vez más patentes las limitaciones de esta forma de estudiar los seres vivos y, de nuevo como en la física, se está empezando a poner el foco en la complejidad de estos sistemas para intentar comprender sus propiedades emergentes.}
    \note[item]{Este sistema tiene un interés biológico inmenso pues la organización de la cromatina dentro del núcleo influye en muchos procesos biológicos, como la transcripción del ADN o la división celular.}
\end{frame}

\subsection{La física de polímeros}

\begin{frame}{Modelos ideales}
    \begin{figure}
        \centering
        \begin{tikzpicture}[scale=1.5]
            \draw[black,semithick] (+0.0,+0.0) -- (+0.9,+0.1);
            \draw[black,semithick] (+0.9,+0.1) -- (+1.2,+0.9);
            \draw[black,semithick] (+1.2,+0.9) -- (+1.9,+1.2);
            \draw[black,semithick] (+1.9,+1.2) -- (+2.6,+1.2);
            \draw[black,semithick] (+2.6,+1.2) -- (+3.2,+0.6);
            \draw[black,semithick] (+3.2,+0.6) -- (+3.4,+0.0);
            \draw[black,semithick] (+3.4,+0.0) -- (+2.8,-0.4);
            \draw[black,semithick] (+2.8,-0.4) -- (+1.9,-0.5);
            \draw[black,semithick] (+1.9,-0.5) -- (+1.6,+0.3);
            \draw[black,semithick] (+1.6,+0.3) -- (+0.9,+0.7);
            \draw[black,semithick] (+0.9,+0.7) -- (+0.3,+1.4);
            \draw[black,semithick] (+0.3,+1.4) -- (-0.1,+0.8);
            \draw[black,semithick] (-0.1,+0.8) -- (-0.9,+0.6);
            \draw[-latex,color1,semithick] (+3.8,+1.6) -- (+2.6,+1.2) node[midway,above]{$\bar{r}_i$};
            \draw[-latex,color1,semithick] (+3.8,+1.6) -- (+4.4,+1.0) node[midway,left]{$\bar{b}_i$};
            \draw[latex-latex,color1,semithick,dashed] (+2.6,+1.2) -- (+3.2,+0.6) node[midway,left]{$l_i$};
            \draw[color1,semithick,dashed] (+2.6,+1.2) -- (+3.3,+1.2);
            \draw[color1,semithick] (+3.1,+1.2) arc (0:-45:0.5) node[midway,right]{$\theta_i$};
            \draw[latex-latex,color1,semithick,dashed] (+2.6,+1.2) -- (+0.3,+1.4) node[midway,above]{$d_{ij}$};
            \draw[black,semithick,fill] (+3.8,+1.6) circle (1pt) node[anchor=south]{$O$};
        \end{tikzpicture}
    \end{figure}
    \begin{block}{Distancia (espacial) promedio}
        \begin{equation}
            d(s)\propto s^{1/2}
        \end{equation}
    \end{block}
    \begin{block}{Probabilidad de contacto}
        \begin{equation}
            P(s)\propto s^{-3/2}
        \end{equation}
    \end{block}
    \note[item]{Como veremos en detalle en el siguiente capítulo la cromatina a nivel biofísico es, ante todo, un polímero y por ello es indispensable hacer una breve introducción a la física de estas macromoléculas antes de adentrarnos en su modelización.}
    \note[item]{Aquí consideraremos que un polímero es sencillamente una colección de $N$ ``partículas'' con posiciones $\{\bar{r}_i\}$, unidas sucesivamente, en un medio a temperatura $T$.}
    \note[item]{A pesar de su sencillez, su elevado número de grados de libertad y las ligaduras entre los mismos hace que el espacio de configuraciones de un polímero sea muy extenso y complejo.}
    \note[item]{Obligatoriamente tenemos que empezar por los llamados modelos ``ideales'', que son aquellos en los que no se tienen en cuenta las interacciones entre monómeros muy separados a lo largo de la cadena (¡aunque estén cerca en el espacio!) y el primero de ellos es el \textit{freely-jointed chain} (FJC). En este modelo no hay interacción alguna entre los monómeros de la cadena, que están unidos por enlaces $\bar{b}_i=\bar{r}_{i+1}-\bar{r}_i$ de longitud fija $l_0$.}
    \note[item]{Aunque visualmente se puedan analizar de forma cualitativa, para caracterizar los distintos modelos cuantitativamente deberemos introducir algunos observables.}
\end{frame}

\begin{frame}
    \begin{block}{Elasticidad}
        \begin{equation}
            V_e=\sum_{i=1}^{N-1}\frac{1}{2}k_e(l_i-l_0)^2
        \end{equation}
    \end{block}
    \begin{block}{Flexibilidad}
        \begin{equation}
            V_b=\sum_{i=1}^{N-2} k_b (1-\cos(\theta_i))
        \end{equation}
    \end{block}
    \begin{block}{Longitud de persistencia}
        \begin{equation}
            l_P=-\frac{l_0}{\ln(\coth(\beta k_b)-1/(\beta k_b))}\approx\beta k_bl_0
        \end{equation}
    \end{block}
    Para $s\ll l_P/l_0$ $d(s)\propto s^{1}$ y $P(s)\propto s^{-3}$.
    \note[item]{Sin embargo, considerar que los enlaces son completamente rígidos es muy idealizado e introduce ligaduras en el modelo que complican su simulación. Por ello es mucho más conveniente y común considerar que los enlaces son extensibles uniendo los monómeros mediante un potencial suave, normalmente un potencial armónico.}
    \note[item]{Además, las configuraciones de este modelo FJC extensibl en condiciones normales son análogas a las de un modelo FJC con enlaces rígidos de longitud $\langle l \rangle=l_0(1+2/(1+\beta k_e l_0^2))$ (donde $\beta=1/k_BT$ y $k_B$ es la constante de Boltzmann) por lo que, especialmente si $k_e$ es elevada, supone una modificación muy pequeña.}
    \note[item]{El siguiente es el modelo de Kratky-Porod o \textit{worm-like chain} discreto (WLC), una evolución del anterior con la que podemos describir polímeros semi-flexibles donde existe una correlación entre las direcciones de enlaces consecutivos.}
\end{frame}

\begin{frame}{Modelos reales}
    \begin{block}{Potencial de Wang-Frenkel}
        \begin{equation}
            V_{WF}=\sum_{i=0}^{N-1}\sum_{j=i+1}^{N-1}\varepsilon\left(\left(\frac{\sigma}{d_{ij}}\right)^{2}-1\right)\left(\left(\frac{2\sigma}{d_{ij}}\right)^{2}-1\right)^{2}
        \end{equation}
    \end{block}
    \begin{block}{Interacción puramente repulsivas}
        \begin{equation}
            d(s)\propto s^{\nu}  \quad \text{y} \quad P(s)\propto s^{-\nu(3+\theta)}
        \end{equation}
        donde $\nu\simeq0.588$ y $\theta\simeq0.71$ luego $\nu(3+\theta)\simeq2.2$.
    \end{block}
    \begin{block}{Interacción atractiva}
        \begin{equation}
            d(s)\propto s^{1/3}  \quad \text{y} \quad P(s)\propto s^{-1}.
        \end{equation}
    \end{block}
    \note[item]{Los modelos que sí que pertenecen a otra clase y tienen leyes de escala con exponentes distintos son los que tienen en cuenta las interacciones entre monómeros muy alejados a lo largo de la cadena. Estas interacciones evidentemente existen en la realidad y son esenciales para describir el comportamiento de muchos polímeros reales, incluyendo por supuesto la cromatina.}
    \note[item]{En particular, siempre existirá al menos una fuerza repulsiva que impida que dos monómeros cualesquiera se superpongan, la llamada interacción de volumen excluido. Para introducirla a nuestro modelo añadiremos, de la multitud de potenciales posibles, un potencial de Wang-Frenkel solo repulsivo.}
    \note[item]{También pueden existir interacciones atractivas entre los monómeros, que de hecho serán muy importantes en la cromatina. Podemos añadir muy fácilmente interacciones de este tipo aumentando el radio de corte del potencial de Wang-Frenkel de modo que tenga la forma de la ecuación anterior para $d_{ij}\leq2\sigma$ y sea $0$ de otro modo.}
    \note[item]{En este caso obtendremos (si $\varepsilon$ es suficientemente grande en comparación con $k_BT$) configuraciones que se llaman de ``glóbulo colapsado'' en las que el polímero ocupa el mínimo espacio posible.}
\end{frame}

\section{La cromatina}

\subsection{Propiedades biofísicas}

\begin{frame}
    \begin{itemize}
        \item La unión del ADN con las proteínas se forma con el principal objetivo de \alert{concentrar} el material genético para que pueda caber dentro del núcleo celular.
        \item Las proteínas principales de la cromatina son las \alert{histonas}, que forman octámeros alrededor de los que se enrolla el ADN formando los llamados \alert{nucleosomas}.
        \item A veces estos nucleosomas se ordenan de forma helicoidal formando una estructura conocida como la \alert{fibra de 30 nm}.
    \end{itemize}
    \begin{figure}
        \centering
        \includegraphics[width=0.4\textwidth]{../Multimedia/Images/Chromatin-Structure.png}
    \end{figure}
    \note[item]{En cada nucleosoma se enrollan $\sim150$ pares de bases y entre un nucleosoma y otro hay $\sim50$ pares de bases de ADN \textit{linker} (al cual se asocia la histona H1 que ayuda a estabilizar la estructura) por lo que se suele considerar que hay un nucleosoma cada $\sim200$ pares de bases.}
    \note[item]{Aunque la geometría exacta de esta estructura está a debate hay unos 11 nucleosomas por cada 11 nm de la fibra, que tiene más exactamente 33 nm de ancho. Pero no toda la cromatina se encuentra en este estado tan compacto, la gran mayoría no tiene una estructura rígida por encima de la escala del nucleosoma.}
\end{frame}

\begin{frame}{La heterogeneidad de la cromatina}
    \begin{itemize}
        \item Este diferente nivel de compactación nos permite distinguir dos tipos de cromatina: la \alert{heterocromatina}, más compacta, y la \alert{eucromatina}, menos compacta.
        \item Este distinto nivel de compactación también se relaciona con una \alert{distinta actividad transcripcional}.
        \item También se pueden distinguir dos tipos de heterocromatina: la \alert{constitutiva} y la \alert{facultativa}.
        \item Los \alert{TADs} son regiones del genoma que interaccionan preferentemente consigo mismas, formando dominios estructurales dentro de la cromatina que se mantienen relativamente independientes de otras regiones.
        \item Los \alert{LADs} son regiones de la cromatina que interactúan (con mayor frecuencia) con la \alert{lámina} nuclear, una red de proteínas ubicada en la parte interior de la envoltura nuclear en la que está confinada la cromatina.
    \end{itemize}
    \note[item]{A menudo las histonas sufren además diversas modificaciones postraduccionales (es decir, cambios químicos ocurridos después de su síntesis por los ribosomas). Estas modificaciones se relacionan con varias propiedades de la cromatina, como por ejemplo la tri-metilación de la novena lisina de la histona H3 que se asocia con frecuencia con la heterocromatina.}
    \note[item]{La eucromatina, que es menos compacta, es más accesible a los factores de transcripción y a otras proteínas que regulan la expresión génica, lo que facilita la transcripción del ADN a ARN.}
    \note[item]{La primera está formada por las regiones de heterocromatina comunes a todas las células de una especie dada.}
    \note[item]{Pero la cromatina se puede clasificar de muchas otras formas. Se pueden identificar regiones de la cromatina que juegan roles específicos en la regulación de la estructura del genoma y la expresión génica y dos de las más importantes son los TADs (Dominios de Asociación Topológica) y los LADs (Dominios Asociados a la Lámina).}
    \note[item]{Estas regiones suelen estar asociadas con la heterocromatina y por lo tanto se caracterizan por ser transcripcionalmente inactivas.}
\end{frame}

\begin{frame}{Mapas de contactos Hi-C experimentales}
    \begin{figure}
        \centering
        \includegraphics[width=0.5\textwidth]{../Multimedia/Images/Hi-C-example.png}
    \end{figure}
    \note[item]{Antes de pasar a la descripción del modelo polímerico debemos introducir la técnica Hi-C, que será relevante más adelante. Este es uno de los métodos experimentales más utilizados para estudiar la organización tridimensional del genoma en el núcleo celular. Esta técnica permite capturar las interacciones físicas entre las diferentes regiones del ADN (es decir, los contactos), revelando cómo se pliega y organiza dentro del espacio nuclear. En Hi-C, el ADN se fija con formaldehído para preservar las interacciones, se fragmenta y se marcan los extremos de los fragmentos cercanos, que luego se ligan para formar moléculas híbridas representativas de las interacciones espaciales. Estas moléculas se secuencian, y los datos obtenidos se analizan para construir mapas de contactos que muestran la frecuencia con la que interaccionan diferentes partes del genoma.}
\end{frame}

\subsection{Modelo polimérico}

\begin{frame}
    \begin{block}{Tipos de partículas}
        \begin{enumerate}
            \item LADh (A): Dominios Asociados a la Lámina heterocromáticos
            \item LNDe (B): Dominios No asociados a la Lámina eucromáticos
        \end{enumerate}
    \end{block}
    \begin{figure}
        \centering
        \includegraphics[scale=0.75]{../Plots/sequence.pdf}
    \end{figure}
    \note[item]{Queda claro por tanto que la cromatina es, en una primera aproximación, un conjunto de polímeros (uno por cromosoma) de gran tamaño confinados dentro de un núcleo aproximadamente esférico. Además tendremos que considerar que hay distintos tipos de monómeros con propiedades locales diferentes.}
    \note[item]{Para empezar consideraremos que cada una de nuestras partículas contiene 33 nucleosomas y un diámetro de 33 nm. Aunque en el estado eucromático los nucleosomas no están organizados ocupan un volumen similar, especialmente dentro de cada TAD.}
    \note[item]{Como podemos asociar los LADs con la heterocromatina, por simplicidad, usaremos sólo dos tipos de partículas.}
    \note[item]{La secuencia exacta de partículas utilizadas en este tipo de modelos no cambia significativamente los resultados. No obstante, basaremos la secuencia de nuestro modelo en datos experimentales. Utilizaremos los LADs del genoma del C. elegans para determinar las partículas LADh de nuestro modelo.}
    \note[item]{En la figura se aprecia que efectivamente hay una correlación entre las regiones heterocromáticas y los LADs. También se ve que en el modelo se reproducen con precisión todos los LADs.}
    \note[item]{Dado que las células del nematodo C. elegans son diploides, dentro del núcleo colocaremos dos copias de cada cromosoma.}
\end{frame}

\begin{frame}{Parámetros del modelo}
    \begin{block}{Parámetros fijos}
        \begin{table}
            \centering
            \begin{tabular}{c c c}
                \toprule
                parámetro       & valor (SI)         & valor (UR)          \\
                \midrule
                $l_0$           & $33$ nm            & $1$ $l_u$           \\
                $k_e$           & $0.48$ pN/nm       & $128$ $e_u$/$l_u^2$ \\
                $k_b^A$         & $4.11$ pN$\cdot$nm & $1$ $e_u$           \\
                $\sigma$        & $33$ nm            & $1$ $l_u$           \\
                $\epsilon^{AA}$ & $2.06$ pN$\cdot$nm & $0.5$ $e_u$         \\
                $\epsilon'$     & $32.9$ pN$\cdot$nm & $8$ $e_u$           \\
                \bottomrule
            \end{tabular}
        \end{table}
    \end{block}
    \begin{block}{Parámetros modificables}
        \begin{enumerate}
            \item El radio del núcleo ($R_n$).
            \item El número de sitios de unión a la lámina ($N'$).
        \end{enumerate}
    \end{block}
    \note[item]{La heterocromatina tiene una longitud de persistencia de $\sim200$ nm lo que en nuestro modelo se traduce en $k_b=6$. Sin embargo, como la heterocromatina sólo representa $\sim$ el $8\%$ de toda la secuencia y pertenece en la mayoría de los casos a algún LAD haremos que las partículas LADh tengan $k_b=1$.}
    \note[item]{Las regiones inactivas tienen una mayor autoafinidad, entre otros motivos, por la atracción por depleción. Afortunadamente podemos hacer una estimación de las fuerzas de interacción entre nuestros dos tipos de partículas gracias a trabajos computacionales anteriores. Consideraremos que las partículas LNDe son puramente repulsivas y las LADh tienen una interacción atractiva de $0.5k_BT$.}
    \note[item]{El valor exacto de la constante elástica no es realmente relevante siempre que sea lo suficientemente alto como para evitar que la distancia entre las partículas enlazadas fluctúe demasiado.}
    \note[item]{Para modelar la atracción de los LADs a la lamina, de forma similar a otros estudios, añadiremos pozos de potencial en posiciones aleatorias de la misma que representan los lugares de la lámina a los que un LAD puede unirse a través de las llamadas proteínas de unión a la lámina. Basándonos en trabajos anteriores haremos que estos pozos tengan una profundidad de $8k_BT$.}
\end{frame}

\section{Metodología}

\subsection{Dinámica Molecular}

\begin{frame}
    \begin{block}{Ecuación de Langevin}
        \begin{equation}
            m_i\frac{d^2\bar{r}_i}{dt^2}=-\xi_i\frac{d\bar{r}_i}{dt}-\bar{\nabla}_iV\left(\{\bar{r}_j\}\right)+\bar{\zeta}_i(t)
        \end{equation}
        donde
        \begin{equation}
            \langle \zeta_i^\alpha(t) \rangle=0 \quad \text{y} \quad \langle \zeta_i^\alpha(t)\zeta_j^\beta(t') \rangle=2\xi_ik_BT\delta_{ij}\delta_{\alpha\beta}\delta(t-t')
        \end{equation}
        y $\xi=3\pi\eta d=0.28$ pg/$\mu$s
    \end{block}
    \begin{block}{Dinámica Browniana}
        $\text{Re}=\rho v_ul_u/\eta=1.7\cdot 10^{-5}$ luego
        \begin{equation}
            \xi_i\frac{d\bar{r}_i}{dt}=-\bar{\nabla}_iV\left(\{\bar{r}_j\}\right)+\bar{\zeta}_i(t)
        \end{equation}
    \end{block}
    \note[item]{En este trabajo realizaremos concretamente simulaciones de dinámica molecular, que son aquellas en las que se resuelven numéricamente las ecuaciones del movimiento clásicas.}
    \note[item]{En la ecuación $m_i$ es la masa de la partícula $i$, $\xi_i$ es su coeficiente cinético y $\bar{\zeta}_i(t)$ es la fuerza aleatoria sobre ella.}
    \note[item]{Pero además la cromatina está inmersa en un solvente, el agua, que tiene efectos importantes sobre el comportamiento del soluto. La forma más sencilla de incorporar sus efectos es la dinámica de Langevin, que consiste simplemente en añadir a las ecuaciones del movimiento de la dinámica molecular una fuerza de fricción proporcional a la velocidad y una fuerza aleatoria que representa el efecto de estas colisiones.}
    \note[item]{Para estimar el valor de $\xi$ nos valdremos de la fórmula de Stokes, donde $\eta$ es la viscosidad del agua y $d$ el diámetro de la partícula (en nuestro caso $\sigma$). Esto significa que estamos en el régimen de bajo número de Reynolds y en la dinámica dominan los efectos disipativos frente a los inerciales.}
\end{frame}

\subsection{Programación en GPUs}

\begin{frame}
    \begin{itemize}
        \item Las \alert{GPUs} son procesadores diseñados para la generación de gráficos 3D pero dadas sus características también son muy apropiadas para propósitos científicos como la realización de \alert{simulaciones}.
        \item La característica de las GPUs (además de su precio) que las hace tan atractivas para estas aplicaciones es su gran \alert{paralelismo}.
        \item Aunque hay muchas herramientas para el desarrollo de este \textit{software} el estándar es \alert{CUDA} (\textit{Compute Unified Device Architecture}): una plataforma de computación en paralelo creada por NVIDIA que permite ejecutar en GPUs de esta empresa programas escritos utilizando CUDA C, una variación del \alert{lenguaje C}.
    \end{itemize}
    \begin{figure}
        \centering
        \includegraphics[width=0.5\textwidth]{../Multimedia/Images/NVIDIA.png}
    \end{figure}
    \note[item]{Este simulaciones tendrán una gran cantidad de partículas ($30386$) y por lo tanto serán muy costosas computacionalmente, lo cual hasta hace poco había impedido estudiar el núcleo celular completo mediante simulaciones numéricas. Por esta razón es necesario aprovechar al máximo la potencia de cálculo del \textit{hardware} del que disponemos.}
    \note[item]{Las simulaciones de dinámica molecular se pueden beneficiar en gran medida del paralelismo ya que, en cada iteración, para hallar las nuevas coordenadas de una cierta partícula no necesitamos las nuevas coordenadas del resto, solo las viejas.}
    \note[item]{Con un programa paralelo de simulación en una GPU podemos actualizar a la vez las coordenadas de una gran cantidad de partículas (aunque quizás no todas) ahorrando un tiempo de cálculo considerable.}
\end{frame}

\subsection{Cálculo del volumen excluido}

\begin{frame}
    \begin{figure}
        \centering
        \begin{tikzpicture}[scale=1.5]
            \draw[color1,fill=color1,fill opacity=0.25] (1.7,1.2) circle (0.5) node[opacity=1] {0};
            \draw[color1,fill=color1,fill opacity=0.25] (2.9,2.1) circle (0.5) node[opacity=1] {1};
            \draw[color1,fill=color1,fill opacity=0.25] (2.1,2.4) circle (0.5) node[opacity=1] {2};
            \draw[color1,fill=color1,fill opacity=0.25] (0.9,2.1) circle (0.5) node[opacity=1] {3};
            \draw[color1,fill=color1,fill opacity=0.25] (2.8,2.8) circle (0.5) node[opacity=1] {4};
            \draw[color1,fill=color1,fill opacity=0.25] (0.6,2.8) circle (0.5) node[opacity=1] {5};
            \foreach \i in {0,...,4} {\draw[black] (\i,0) -- (\i,4); \draw[black] (0,\i) -- (4,\i);}
            \foreach \i in {0,...,3} {\foreach \j [evaluate=\num using int(\i+4*\j)] in {0,...,3} {\node[above right] at (\i,\j) {\num};}}
            \draw[latex-latex,color1,semithick,dashed] (0.5,0.0) -- (0.5,1.0) node[midway,right] {$2\sigma$};
        \end{tikzpicture}
    \end{figure}
    \begin{enumerate}
        \item A = [(0,5), (1,10), (2,10), (3,8), (4,10), (5,8)]
        \item A = [(0,5), (3,8), (5,8), (1,10), (2,10), (4,10)]
        \item B = [(0,0), (0,0), (0,0), (0,0), (0,0), (0,1), (1,1), (1,1), (1,3), (3,3), (3,6), (6,6), (6,6), (6,6), (6,6), (6,6)]
    \end{enumerate}
    \note[item]{Como la interacción de volumen excluido será la más costosa computacionalmente creemos conveniente mostrar el algoritmo que se ha empleado para el cálculo de las fuerzas debidas a la misma.}
    \note[item]{La idea clave detrás del algoritmo que vamos a utilizar es que se puede acelerar mucho la búsqueda de las partículas vecinas subdividiendo el espacio de simulación. En concreto utilizaremos una malla uniforme, es decir, subdividiremos el espacio de simulación en celdas del mismo tamaño.}
    \note[item]{Utilizaremos una malla en la que el tamaño de las celdas es $2\sigma$, el radio de corte del potencial de Wang-Frenkel atractivo, de forma que para calcular las fuerzas sobre una partícula solo deberemos calcular la distancia con las partículas que se encuentran en las 27 celdas vecinas, en lugar de todas las partículas de la simulación.}
    \note[item]{Para ilustrar el proceso con el que generaremos la lista de las partículas que hay en cada celda consideremos el caso sencillo en que se representa en la figura. Empezaremos por crear una lista A de tuplas que contengan los índices de cada partícula y los índices de la celda en la que se encuentran. A continuación ordenaremos esta lista en función del índice de la celda. Finalmente recorreremos esta lista y almacenaremos en otra lista B el principio y final de cada celda en la lista A ordenada.}
\end{frame}

\begin{frame}{Rendimiento}
    \begin{figure}
        \centering
        \includegraphics[scale=0.75]{../Plots/performance.pdf}
    \end{figure}
    \note[item]{Todo este proceso se repetirá con cada paso de tiempo y su parte más costosa computacionalmente será la ordenación de la lista A que sin embargo es $\mathcal{O}(N)$ y por lo tanto es mucho más veloz que la implementación \textit{naive} (que sería $\mathcal{O}(N^2)$)para los valores de $N$ elevados que vamos a utilizar. En la figura se ve como efectivamente el tiempo de ejecución del programa completo es proporcional a $N$ (más una constante debido a otros factores que impiden tiempos menores con valores de $N$ bajos).}
\end{frame}

\section{Resultados: organización espacial}

\subsection{Configuración convencional}

\begin{frame}{Visualización de un corte ecuatorial}
    \begin{figure}
        \centering
        \includegraphics[width=0.5\textwidth]{../Multimedia/Images/Conventional-QS.png}
    \end{figure}
    Hemos asignado un color distinto a cada tipo de partícula: el rojo a las LADh, el azul a las LNDe y el gris a los sitios de unión a la lámina.
    \note[item]{Para estudiar el modelo de cromatina que hemos desarrollado realizaremos simulaciones en diversas condiciones variando el radio del núcleo y el número de sitios de unión a la lámina.}
    \note[item]{Además de los observables ya presentados anteriormente, $d(s)$ y $P(s)$, calcularemos $\rho(r)$, la densidad de cromatina en función de la distancia al centro del núcleo, y $P(i,j)$, el mapa de contactos (como el que se obtiene experimentalmente mediante la técnica Hi-C).}
    \note[item]{Para nuestra primera simulación daremos a los parámetros libres valores razonables en condiciones normales. Los núcleos de C. elegans tienen un radio de aproximadamente $1$ $\mu$m luego la densidad de cromatina promedio dentro del núcleo es $\sim 0.2$. Como lo habitual es que los LADs puedan adherirse a cualquier parte de la lámina haremos que estos ocupen una fracción lo más elevada posible de la superficie del núcleo.}
    \note[item]{De esta forma obtenemos configuraciones como la que se puede ver en esta figura. Vemos claramente que en esta ocasión las partículas LADh se colocan en el exterior del núcleo, cercanas a la lámina, y las LNDe ocupan el resto del núcleo de forma bastante homogénea. Esta es la configuración que se observa experimentalmente en la grandísima mayoría de células por lo que la llamaremos la configuración ``convencional''.}
\end{frame}

\begin{frame}{Densidad radial de cromatina}
    \begin{figure}
        \centering
        \includegraphics[scale=0.75]{../Plots/30386-0.2-0.4-0.0-0.0/rcd.pdf}
    \end{figure}
    \note[item]{En esta figura se representa la densidad de la cromatina (tanto la de ambos tipos de partícula por separado como la total) y podemos ver esta separación de fases también de forma muy clara. Podemos apreciar también que, como resultado de las interacciones atractivas de las partículas LADh consigo mismas y con la lámina, la densidad de cromatina (en concreto de partículas LADh) cerca de la lámina es mucho mayor que en el resto del núcleo. Esta es una propiedad que también se observa experimentalmente y que ayuda a reprimir la transcripción de estas partes del genoma pues dificulta la difusión de los factores de transcripción dentro de estas regiones.}
\end{frame}

\begin{frame}{Distancia (espacial) promedio entre monómeros}
    \begin{figure}
        \centering
        \includegraphics[scale=0.75]{../Plots/30386-0.2-0.4-0.0-0.0/sd.pdf}
    \end{figure}
    \note[item]{Para analizar los dos siguientes observables hemos obtenido el exponente con el que escalan en diversos rangos de $s$ ajustando las curvas de cada cromosoma a leyes de potencias y promediando los parámetros de estos ajustes sobre todos los cromosomas excepto el X, el cromosoma sexual, porque tiene una proporción de partículas LADh muy inferior a los autosomas.}
    \note[item]{En esta figura vemos que para $s$ muy bajo el exponente es $\sim0.7$, algo superior al de un modelo con volumen excluido ($\sim0.588$), como consecuencia de la longitud de persistencia de las partículas LADh que a escalas pequeñas es relevante. Después encontramos una región de $s$ en la que el exponente es $\sim0.5$ como en los modelos ideales, pero que es muy pequeña por lo que estos modelos no sirven para describir la cromatina. En cambio para un rango grande de $s$ vemos que el exponente es $\sim0.3$ (que es compatible con los resultados experimentales) luego el modelo de ``glóbulo colapsado'' sí que es una descripción adecuada. Para $s$ cercano a la longitud total de los cromosomas vemos que $d(s)$ prácticamente se detiene pues la cromatina está confinada. La distancia media entre los puntos de de una esfera es $(36/35)R$, pero $d(s)$ se detiene en los $0.6$ $\mu$m que es bastante inferior a $\sim1$ $\mu$m. Esto es una demostración más de los territorios cromáticos.}
\end{frame}

\begin{frame}{Probabilidad de contacto}
    \begin{figure}
        \centering
        \includegraphics[scale=0.75]{../Plots/30386-0.2-0.4-0.0-0.0/cp.pdf}
    \end{figure}
    \note[item]{En esta figura en cambio distinguimos, de forma bastante clara, solo dos régimenes. En el primero el exponente con el que decae $P(s)$ es $\sim-2.4$, algo inferior al predicho por la teoría para un modelo FJC con volumen excluido, de nuevo debido a la longitud de persistencia de las partículas LADh. En el segundo, que representa un rango muy grande de valores de $s$, observamos que el exponente es $\sim-0.9$ (que también es compatible con los resultados experimentales), muy similar al que esperaríamos en un ``glóbulo colapsado'', por lo que concluimos otra vez que este es, de los modelos teórico que hemos visto, el que mejor describe la cromatina.}
    \note[item]{Como anticipábamos en ambos observables (pero especialmente en $P(s)$) se ve que el cromosoma X, debido a que en proporción tiene muchas más partículas LNDe, se comporta de forma distinta a los demás y adopta configuraciones más extendidas por lo que $d(s)$ es mayor y $P(s)$ es menor en este cromosoma.}
\end{frame}

\begin{frame}{Mapa de contactos}
    \begin{figure}
        \centering
        \includegraphics[scale=0.60]{../Plots/30386-0.2-0.4-0.0-0.0/cm.pdf}
    \end{figure}
    \note[item]{Finalmente en esta figura mostramos el mapa de contactos que también nos aporta información muy interesante. Vemos primero de todo la clarísima separación de los distintos territorios cromáticos que, a pesar de la longitud de la simulación, sigue siendo muy marcada. Esto está en concordancia con los experimentos y con diversos estudios teóricos. A pesar de que la configuración de equilibrio no tiene territorios cromáticos, el tiempo de relajación del sistema, es decir el tiempo que se espera que haga falta para que los cromosomas se mezclen es de años.}
    \note[item]{El mapa de contactos también nos muestra la separación de fases dentro de cada cromosoma mediante un patrón de tablero de ajedrez. Como en los autosomas de esta especie la heterocromatina se concentra en los extremos en el mapa de contactos de cada uno se genera una cuadrícula $3\times3$ en la que las esquinas, que corresponden a interacciones LADh-LADh, son las más intensas, y el cuadrado central, que representa interacciones LNDe-LNDe, es menos intenso que las esquinas pero más que el resto de regiones, que representan interacciones cruzadas. Esto es una consecuencia clara de la separación de fases, que lleva a las partículas LNDe a interactuar consigo mismas con mayor frecuencia a pesar de que no exista una interacción atractiva entre ellas.}
\end{frame}

\subsection{Configuración invertida}

\begin{frame}{Visualización de un corte ecuatorial}
    \begin{figure}
        \centering
        \includegraphics[width=0.5\textwidth]{../Multimedia/Images/Inverted-QS.png}
    \end{figure}
    \note[item]{Con la siguiente simulación reproduciremos otro tipo de configuración que también se observa en algunas células y analizaremos la función de la lámina en la organización de la cromatina pues vamos a eliminar los sitios de unión a la lámina sin variar el radio del núcleo.}
    \note[item]{Vemos en la figura que de esta forma se obtienen configuraciones que denominaremos invertidas pues las partículas LNDe y LADh se intercambian posiciones. Esta peculiar configuración se observa en las células de los bastones de los animales nocturnos pues la aglomeración de la heterocromatina en un único dominio con una posición central alejada de la lámina reduce la dispersión de la luz que lo atraviesa, mejorando la visión nocturna de estos animales.}
\end{frame}

\begin{frame}{Densidad radial de cromatina}
    \begin{figure}
        \centering
        \includegraphics[scale=0.75]{../Plots/30386-0.2-0.0-0.0-0.0/rcd.pdf}
    \end{figure}
    \note[item]{En esta figura también observamos que se han invertido las posiciones típicas de ambos tipos de partículas. Sin embargo, también vemos que hay mucho más ruido y las barras de error son mucho mayores, es decir, que las configuraciones obtenidas varían mucho más con las condiciones iniciales y fluctuan mucho más en el tiempo. Esto significa que la lámina ayuda a estabilizar la organización de la cromatina. Esta estabilidad es probablemente una de las múltiples razones por las que en la gran mayoría de células la cromatina interacciona con la lámina y adopta la configuración convencional. También debemos resaltar que en esta gráfica parecería que las partículas LADh no ocupan el centro exacto del núcleo. Sin embargo, la gran incertidumbre en los valores de $\rho(r)$ cerca del centro nos indica que falta estadística y lo más esperable es que, prolongando las simulaciones, vieramos que la densidad de este tipo de partículas desciende monótonamente con $r$.}
    \note[item]{Estos resultados nos indican que la lámina es indispensable para que la cromatina se organice de la forma convencional. Existen fuerzas entrópicas que favorecen que la heterocromatina se coloque cerca de la pared del núcleo, pero cuando se utilizan valores realistas para los parámetros del modelo, estas no son suficientemente fuertes y se obtienen configuraciones invertidas.}
\end{frame}

\begin{frame}{Mapa de contactos}
    \begin{figure}
        \centering
        \includegraphics[scale=0.60]{../Plots/30386-0.2-0.0-0.0-0.0/cm.pdf}
    \end{figure}
    \note[item]{Por otro lado los observables $d(s)$ y $P(s)$ apenas cambian con respecto a la configuración convencional, por lo que no mostraremos las gráficas de estos dos observables en esta ocasión. La única diferencia que se puede apreciar es que el cromosoma X está algo más extendido, es decir, que en comparación $d(s)$ es algo mayor y $P(s)$ es algo menor, probablemente por la sencilla razón de que como veremos a continuación todos los cromosomas tienen algo más de movilidad.}
    \note[item]{Finalmente, en el mapa de contactos representado en esta figura vemos los mismos patrones que antes, consecuencia de la separación de fases, pero además vemos claramente que hay muchos más contactos entre cromosomas distintos: los territorios cromáticos son mucho más difusos. Esto nos demuestra que la lámina ayuda a mantener durante más tiempo los territorios cromáticos, como ya habían observado otros estudios computacionales, y este hecho es sin lugar a dudas otra de las razones por las que la configuración convencional es mucho más común.}
\end{frame}

\section{Conclusiones y trabajo futuro}

\begin{frame}
    \begin{figure}
        \centering
        \includegraphics[width=1.00\textwidth]{../Multimedia/Images/decay-length.png}
    \end{figure}
\end{frame}

\begin{frame}
    \begin{figure}
        \centering
        \includegraphics[width=0.75\textwidth]{../Multimedia/Images/bleb.png}
    \end{figure}
\end{frame}

\end{document}
