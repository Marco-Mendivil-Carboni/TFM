\chapter{Resultados: organización espacial}
\label{cap:results}

Para estudiar el modelo de cromatina que hemos presentado en el capítulo \ref{cap:chromatin} realizaremos simulaciones en diversas condiciones variando los parámetros $R_n$ y $N'$ pero antes de mostrar los resultados de las mismas debemos explicar las condiciones iniciales que hemos utilizado y presentar los observables que nos ayudarán a caracterizar la configuración de la cromatina.

La condición inicial no altera los resultados de una simulación de dinámica molecular si se alcanza el equilibrio pero pronto veremos que el núcleo celular no se encuentra en el equilibrio por lo que será muy importante que las condiciones iniciales tengan sentido en el contexto de nuestro problema. En concreto consisitirán en doce caminatas aleatorias autoexcluyentes (una por cada cromosoma) dentro de doce subdivisiones iguales de la esfera del núcleo. De esta forma pretendemos introducir en nuestras simulaciones la separación de la cromatina en territorios cromáticos que se observa experimentalmente y que es consecuencia simplemente de que el estado de la cromatina durante la interfase proviene de la decondensación de los cromosomas en el estado mitótico, que están claramente separados. Esto también hará que debamos repetir cada simulación varias veces para poder promediar sobre las distintas condiciones iniciales, de la misma forma que los datos experimentales suelen obtenerse como el promedio sobre distintas células. Sin embargo veremos que la mayoría de observables no cambian excesivamente con la condición inicial por lo que repetir $8$ veces cada simulación será suficiente. Por otro lado prolongaremos cada simulación hasta que represente $1.2$ s de tiempo real, lo que equivaldrá a más de $3\cdot10^7$ iteraciones del método de Runge-Kutta.

Además de los observables ya presentados en el capítulo \ref{cap:introduction}, $d(s)$ y $P(s)$, calcularemos $\rho(r)$, la densidad de cromatina en función de la distancia al centro del núcleo, y el mapa de contactos como los que se obtienen experimentalmente mediante la técnica Hi-C. Para calcular $\rho(r)$ subdividiremos el núcleo en $64$ bines (coronas esféricas) del mismo volumen y hallaremos la fracción de ese volumen que está ocupada por partículas de cromatina. Para calcular el mapa de contactos detectaremos los contactos entre partículas de la misma forma que para calcular $P(s)$ pero almacenaremos por separado los contactos de cada pareja de partículas en lugar de promediar sobre todas las parejas de partículas con la misma separación a lo largo de la cadena.

\section{Configuración convencional}

Para nuestra primera simulación daremos a los parámetros libres valores razonables en condiciones normales. Los núcleos de C. elegans tienen un radio de aproximadamente $1$ $\mu$m \cite{Ikegami2010}, lo que significa que la densidad de cromatina promedio dentro del núcleo es $\sim 0.2$. Resulta díficil estimar la cantidad de sitios de unión a la lámina que debería haber ya que está es una descripción muy mesoscópica de la lámina pero como lo habitual es que los LADs puedan adherirse a cualquier parte de la lámina haremos que estos ocupen una fracción elevada de la superficie del núcleo, en concreto $0.4$ (por lo que $N'=5848$) ya que valores superiores son díficiles de alcanzar colocando estos sitios de forma aleatoria y es más que suficiente para que estén repartidos de forma homogénea por toda la superficie.

\begin{figure}
    \centering
    \includegraphics[width=0.5\textwidth]{../Multimedia/Images/Conventional-QS.png}
    \caption{Visualización de una configuración convencional de nuestro modelo de la cromatina.}
    \label{fig:vmd_QS_image_c}
\end{figure}

De esta forma obtenemos configuraciones como la que se puede ver en la figura \ref{fig:vmd_QS_image_c}, que se ha obtenido mediante el programa de visualización de simulaciones moleculares VMD \cite{Humphrey96}. En esta imagen mostramos solo una porción de la cromatina para poder ver la distribución de la misma en el interior del núcleo y, en lugar de representar cada una de las partículas de forma individual, hemos utilizado una forma de representación ``suavizada'' que nos ofrece VMD. En este modo de representación se genera una superficie que rodea a las partículas que se encuentran muy cercanas para facilitar la interpretación visual rápida de las imágenes de este tipo. Además hemos asignado un color distinto a cada tipo de partícula: el rojo a las LADh, el azul a las LNDe y el gris a los sitios de unión a la lámina.

La imagen corresponde al último frame de la primera condición inicial.

Vemos claramente que en esta ocasión las partículas LADh se colocan en el exterior como se ve experimentalmente en la grandísima mayoría de células.

\begin{figure}
    \centering
    \includegraphics{../Plots/Simulations/30386-0.200-0.400-0.000-0.000/rcd.pdf}
    \caption{\dots}
    \label{fig:rcd_c}
\end{figure}

Hay separación de fases como se ve en rcd.

\begin{figure}
    \centering
    \includegraphics{../Plots/Simulations/30386-0.200-0.400-0.000-0.000/sd.pdf}
    \caption{\dots}
    \label{fig:sd_c}
\end{figure}

MENCIONAR QUE LA DISTANCIA PROMEDIO ENTRE PUNTOS ALEATORIOS EN UNA ESFERA DE RADIO R ES 36/35 R. NO LLEGA PORQUE EXISTEN TERRITORIOS CROMÁTICOS.

Primero sd escala con 0.7 que es un promedio entre 1 y 0.6, luego en un rango muy pequeño escala con 0.6 y luego con 0.3.

Es decir, el rango en que la descripción de polímero ideal es válida es muy pequeña y en la mayoría de escalas se comporta como un polímero colapsado y, finalmente confinado.

\begin{figure}
    \centering
    \includegraphics{../Plots/Simulations/30386-0.200-0.400-0.000-0.000/cp.pdf}
    \caption{\dots}
    \label{fig:cp_c}
\end{figure}

cp nos cuenta un poco lo mismo, vemos que decae con -2.4, muy cercano al 2.2 predicho por la teoría y luego con -0.9 muy cercano al -1 de un polímero colapsado.

En estas dos figuras los ajustes los hemos realizado sobre las curvas de todos los cromosomas excepto del X, que al tener una proporción de LADh mucho menor, ver figura secuencia, sigue una tendencia claramente separada.

En el cromosoma X los contactos son menos frecuentes y las distancias mayores como es de esperar pero los exponentes son prácticamente los mismos.

\begin{figure}
    \centering
    \includegraphics{../Plots/Simulations/30386-0.200-0.400-0.000-0.000/cm.pdf}
    \caption{\dots}
    \label{fig:cm_c}
\end{figure}

En el mapa de contactos también se recoge información muy interesante. Vemos primero de todo la clarísima separación en territorios cromáticos que, a pesar de la longitud de la simulación sobrevive y sigue muy marcada. Esto está en concordancia con los experimentos porque, a pesar de que la configuración de equilibrio se ha demostrado teoricamente que no tiene territorios cromáticos, el tiempo que se espera que haga falta para que los polímeros se difundan (empezando por sus extremos) y se mezclen es de siglos y evidentemente los ciclos celulares son mucho más cortos. Es decir que esta es una característica que no necesita ningún mecanismo que la mantenga de forma activa sino que sobrevive por la simple lentitud de la dinámica de un polímero de tal magnitud.

Los mapas de contactos también nos muestran de forma cristalina la separación de fase dentro de cada cromosoma. Una ventaja de haber utilizado la secuencia del c elegans es que esta tiene en todos los autosomas (todos los cromosomas menos el sexual, el X) la heterochromatina se concentra en los extremos de forma que se en el mapa de contactos se genera una cuadrícula 3x3 en las que las esquinas, que corresponden a interacciones LADh-LADh son las más intensas, como es lógico, pero además el cuadrado central, que representa interacciones LNDe-LNDe, es menos intenso que los primeros pero más que los demás, que represntan interacciones cruzadas. Esto es una consecuencia clara de la separación de fases que lleva a las partículas LNDe a interactuar consigo mismas con mayor frecuencia a pesar de no haber introducido una interacción atractiva entre las mismas.

\section{Configuración invertida}

La siguiente simulación que hemos decidido hacer es para analizar la función de la lámina (o más bien de las interacciones con la misma) y para reproducir otro tipo de configuración que también se observa en algunas células, MIRAR CUALES Y CITAR. Para ello eliminamos los sitios de unión a la lámina dejando el radio igual.

\begin{figure}
    \centering
    \includegraphics[width=0.5\textwidth]{../Multimedia/Images/Inverted-QS.png}
    \caption{Visualización de una configuración invertida de nuestro modelo de la cromatina.}
    \label{fig:vmd_QS_image_i}
\end{figure}

Vemos que de esta forma a pesar de lo que decían Cook y marenduzzo CITAR cuando se introducen los valores reales de longitud de persistencia en proporciones y densidades realistas las fuerzas entrópicas que favorecen que la heterochromatina se coloque en el exterior no vencen a la atracción que tiene consigo misma y se condensa alejandose de la lámina y ocupando una posición más bien central. Por ello llamaremos a esta configuración invertida.

\begin{figure}
    \centering
    \includegraphics{../Plots/Simulations/30386-0.200-0.000-0.000-0.000/rcd.pdf}
    \caption{\dots}
    \label{fig:rcd_i}
\end{figure}

Este resultado se observa también la rcd donde sin embargo también vemos que las barras de error son mucho mayores, es decir, las configuraciones varían mucho más según las condiciones iniciales y fluctuan mucho más en el tiempo. Esto quiere decir que la lámina ayuda a estabilizar la conformación de la cromatina y esta es probablemente una de las razones que ha llevado a la gran mayoría de células a adoptar la configuración convencional. Además en el histograma se ve que, aunque la heterochromatina está claramente separada de la lámina, a la cual se acerca preferentemente la eucromatina como este glóbulo colapsado de LADs no siempre ocupa el centro exacto del núcleo en el centro del histograma la tendencia se invierte. Pero esto no es real y simplemente quiere decir que habría que alargar las simulaciones más para darle tiempo a este glóbulo de LADs de difundirse.

En sd en está ocasión obtenemos exactamente lo mismo pero el cromosoma X está un poco más separado probablemente por la sencilla razón de que, como ahora veremos, todos los cromosomas tiene algo más de movilidad. cp también es igual. Por ello no mostramos ninguno de los dos.

\begin{figure}
    \centering
    \includegraphics{../Plots/Simulations/30386-0.200-0.000-0.000-0.000/cm.pdf}
    \caption{\dots}
    \label{fig:cm_i}
\end{figure}

El mapa de contactos, además de reproducir la misma separación de fases que antes, vemos claramente que hay más contactos entre cromosomas distintos, es decir, se están difuminando los territorios cromáticos. Es decir la lámina ayuda a mantenerlos! Y esta es otra razón por la que probablemente la configuración convencional es mucho más común.

\section{Longitud de decaimiento}

\dots
