\chapter{Resultados: organización espacial}
\label{cap:results}

\dots

\section{Configuración convencional}

\dots

\begin{figure}
  \centering
  \includegraphics{../Plots/Simulations/30386-0.200-0.400-0.000-0.000/rcd.pdf}
  \caption{\dots}
  \label{fig:rcd_c}
\end{figure}

\begin{figure}
  \centering
  \includegraphics{../Plots/Simulations/30386-0.200-0.400-0.000-0.000/sd.pdf}
  \caption{\dots}
  \label{fig:sd_c}
\end{figure}

MENCIONAR QUE LA DISTANCIA PROMEDIO ENTRE PUNTOS ALEATORIOS EN UNA ESFERA DE RADIO R ES 36/35 R. NO LLEGA PORQUE EXISTEN TERRITORIOS CROMÁTICOS.

\begin{figure}
  \centering
  \includegraphics{../Plots/Simulations/30386-0.200-0.400-0.000-0.000/cp.pdf}
  \caption{\dots}
  \label{fig:cp_c}
\end{figure}

\begin{figure}
  \centering
  \includegraphics{../Plots/Simulations/30386-0.200-0.400-0.000-0.000/cm.pdf}
  \caption{\dots}
  \label{fig:cm_c}
\end{figure}

\section{Configuración invertida}

\dots

\begin{figure}
  \centering
  \includegraphics{../Plots/Simulations/30386-0.200-0.000-0.000-0.000/rcd.pdf}
  \caption{\dots}
  \label{fig:rcd_i}
\end{figure}

\begin{figure}
  \centering
  \includegraphics{../Plots/Simulations/30386-0.200-0.000-0.000-0.000/sd.pdf}
  \caption{\dots}
  \label{fig:sd_i}
\end{figure}

\begin{figure}
  \centering
  \includegraphics{../Plots/Simulations/30386-0.200-0.000-0.000-0.000/cp.pdf}
  \caption{\dots}
  \label{fig:cp_i}
\end{figure}

\begin{figure}
  \centering
  \includegraphics{../Plots/Simulations/30386-0.200-0.000-0.000-0.000/cm.pdf}
  \caption{\dots}
  \label{fig:cm_i}
\end{figure}

\section{Longitud de decaimiento}

\dots

\section{Blebs}

DEJAR ESTO COMO TRABAJO FUTURO. SACAR UNA IMAGEN.

\dots
