\chapter{Resultados: organización espacial}
\label{cap:results}

\dots

\section{Configuración convencional}

\dots

\begin{figure}
  \centering
  \includegraphics{../Plots/Simulations/30386-0.200-0.400-0.000-0.000/rcd.pdf}
  \caption{\dots}
  \label{fig:rcd}
\end{figure}

\begin{figure}
  \centering
  \includegraphics{../Plots/Simulations/30386-0.200-0.400-0.000-0.000/sd.pdf}
  \caption{\dots}
  \label{fig:sd}
\end{figure}

MENCIONAR QUE LA DISTANCIA PROMEDIO ENTRE PUNTOS ALEATORIOS EN UNA ESFERA DE RADIO R ES 36/35 R. NO LLEGA PORQUE EXISTEN TERRITORIOS CROMÁTICOS.

\begin{figure}
  \centering
  \includegraphics{../Plots/Simulations/30386-0.200-0.400-0.000-0.000/cp.pdf}
  \caption{\dots}
  \label{fig:cp}
\end{figure}

\begin{figure}
  \centering
  \includegraphics{../Plots/Simulations/30386-0.200-0.400-0.000-0.000/cm.pdf}
  \caption{\dots}
  \label{fig:cm}
\end{figure}

\section{Configuración invertida}

\dots

\section{Longitud de decaimiento}

\dots

\section{Blebs}

\dots
