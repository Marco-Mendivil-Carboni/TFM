\chapter{Resultados: organización espacial}
\label{cap:results}

Para estudiar el modelo de cromatina que hemos presentado en el capítulo \ref{cap:chromatin} realizaremos simulaciones en diversas condiciones variando el radio del núcleo y el número de sitios de unión a la lámina pero antes de mostrar los resultados de las mismas debemos explicar las condiciones iniciales que hemos utilizado y presentar los observables que nos ayudarán a caracterizar la configuración de la cromatina.

La condición inicial no altera los resultados de una simulación de dinámica molecular si se alcanza el equilibrio pero pronto veremos que el núcleo celular no se encuentra en el equilibrio por lo que será muy importante que las condiciones iniciales tengan sentido en el contexto de nuestro problema. En concreto consisitirán en doce caminatas aleatorias autoexcluyentes (una por cada cromosoma) dentro de doce subdivisiones iguales de la esfera del núcleo. De esta forma pretendemos introducir en nuestras simulaciones la separación de la cromatina en territorios cromáticos que se observa experimentalmente y que es consecuencia simplemente de que el estado de la cromatina durante la interfase proviene de la decondensación de los cromosomas en el estado mitótico, que están claramente separados. Esto también hará que debamos repetir cada simulación varias veces para poder promediar sobre las distintas condiciones iniciales, de la misma forma que los datos experimentales suelen obtenerse como el promedio sobre distintas células. Sin embargo veremos que la mayoría de observables no cambian excesivamente con la condición inicial por lo que repetir $8$ veces cada simulación será suficiente. Por otro lado prolongaremos cada simulación hasta que represente $1.2$ s de tiempo real, lo que equivaldrá a más de $3\cdot10^7$ iteraciones del método de Runge-Kutta.

Además de los observables ya presentados en el capítulo \ref{cap:introduction}, $d(s)$ y $P(s)$, calcularemos $\rho(r)$, la densidad de cromatina en función de la distancia al centro del núcleo, y $P(i,j)$, el mapa de contactos (como el que se obtiene experimentalmente mediante la técnica Hi-C). Para calcular $\rho(r)$ subdividiremos el núcleo en $64$ bines (coronas esféricas) del mismo volumen y hallaremos la fracción de ese volumen que está ocupada por partículas de cromatina. Para calcular el mapa de contactos detectaremos los contactos entre partículas de la misma forma que para calcular $P(s)$ pero almacenaremos por separado los contactos de cada pareja de partículas en lugar de promediar sobre todas las parejas de partículas con la misma separación a lo largo de la cadena.

\section{Configuración convencional}

Para nuestra primera simulación daremos a los parámetros libres valores razonables en condiciones normales. Los núcleos de C. elegans tienen un radio de aproximadamente $1$ $\mu$m \cite{Ikegami2010}, lo que significa que la densidad de cromatina promedio dentro del núcleo es $\sim 0.2$. Resulta díficil estimar la cantidad de sitios de unión a la lámina que debería haber ya que está es una descripción muy mesoscópica de la lámina pero como lo habitual es que los LADs puedan adherirse a cualquier parte de la lámina haremos que estos ocupen una fracción elevada de la superficie del núcleo. En concreto haremos que esta fracción sea $0.4$ (por lo que $N'=5848$) ya que valores superiores son díficiles de alcanzar colocando estos sitios de forma aleatoria y es más que suficiente para que estén repartidos de forma homogénea por toda la superficie.

\begin{figure}[t]
    \centering
    \includegraphics[width=0.5\textwidth]{../Multimedia/Images/Conventional-QS.png}
    \caption{Visualización de una configuración convencional de nuestro modelo de la cromatina.}
    \label{fig:vmd_QS_image_c}
\end{figure}

De esta forma obtenemos configuraciones como la que se puede ver en la figura \ref{fig:vmd_QS_image_c} \footnote{En particular esta imagen representa la última configuración alcanzada en la primera simulación.}, que se ha obtenido mediante el programa de visualización de simulaciones moleculares VMD \cite{Humphrey96}. En esta imagen mostramos solo una porción de la cromatina para poder ver la distribución de la misma en el interior del núcleo y, en lugar de representar cada una de las partículas de forma individual, hemos utilizado una forma de representación ``suavizada'' que nos ofrece VMD. En este modo de representación se genera una superficie que rodea a las partículas que se encuentran muy cercanas para facilitar la interpretación visual rápida de las imágenes de este tipo. Además hemos asignado un color distinto a cada tipo de partícula: el rojo a las LADh, el azul a las LNDe y el gris a los sitios de unión a la lámina.

Vemos claramente que en esta ocasión las partículas LADh se colocan en el exterior del núcleo, cercanas a la lámina, y las LNDe ocupan el resto del núcleo de forma bastante homogénea. Esta es la configuración que se observa experimentalmente en la grandísima mayoría de células \cite{Camara2023} por lo que la llamaremos la configuración convencional.

\begin{figure}[t]
    \centering
    \includegraphics{../Plots/Simulations/30386-0.200-0.400-0.000-0.000/rcd.pdf}
    \caption{Densidad de la cromatina en función de la distancia al centro del núcleo en la configuración convencional.}
    \label{fig:rcd_c}
\end{figure}

En la figura \ref{fig:rcd_c} se representa la densidad de la cromatina (tanto la de ambos tipos de partícula por separado como la total) y podemos ver esta separación de fases también de forma muy clara. Podemos apreciar también que, como resultado de las interacciones atractivas de las partículas LADh consigo mismas y con la lámina, la densidad de cromatina (en concreto de partículas LADh) cerca de la lámina es mucho mayor que en el resto del núcleo. Esta es una propiedad que también se observa experimentalmente y que ayuda a reprimir la transcripción de estas partes del genoma pues dificulta la difusión de los factores de transcripción dentro de estas regiones.

\begin{figure}[p]
    \centering
    \includegraphics{../Plots/Simulations/30386-0.200-0.400-0.000-0.000/sd.pdf}
    \caption{Distancia (espacial) promedio entre dos monómeros en función de su separación a lo largo de la cadena en la configuración convencional.}
    \label{fig:sd_c}
\end{figure}

\begin{figure}[p]
    \centering
    \includegraphics{../Plots/Simulations/30386-0.200-0.400-0.000-0.000/cp.pdf}
    \caption{Probabilidad de que dos monómeros estén en contacto en función de su separación a lo largo de la cadena en la configuración convencional.}
    \label{fig:cp_c}
\end{figure}

A continuación en las figuras \ref{fig:sd_c} y \ref{fig:cp_c} podemos ver respectivamente $d(s)$ y $P(s)$, los observables que vamos a emplear para comparar las configuraciones de nuestro modelo de la cromatina con las de los modelos que introdujimos en el capítulo \ref{cap:introduction}. Para analizarlos hemos obtenido el exponente con el que escalan estos observables en diversos rangos de $s$ ajustando las curvas de cada cromosoma a leyes de potencias y promediando los parámetros de estos ajustes sobre todos los cromosomas excepto el X. Este cromosoma, el cromosoma sexual, tiene una proporción de partículas LADh muy inferior a los autosomas, el resto de cromosomas, y por lo tanto no se comporta exactamente de la misma forma (aunque sus exponentes son muy similares).

En la figura \ref{fig:sd_c} podemos ver que para los valores de $s$ más bajos el exponente de $d(s)$ es $\sim0.7$, algo superior al que se espera de un modelo FJC con volumen excluido ($\sim0.588$), como consecuencia de la longitud de persistencia de las partículas LADh que a estas escalas tan pequeñas es muy relevante. Cuando la longitud de la cadena supera esta longitud de persistencia encontramos una región de valores $s$ en la que el exponente es $\sim0.5$ como en los modelos ideales, pero esta región es muy pequeña por lo que podemos decir que este tipo de modelos no sirven para describir la cromatina excepto en una escala muy concreta. En cambio para un rango bastante grande de valores de $s$ vemos que el exponente es $\sim0.3$ (que es compatible con los resultados experimentales \cite{Wang2016}) lo que significa que el modelo de ``glóbulo colapsado'' sí que es una descripción adecuada de la organización de la cromatina. Finalmente, para valores de $s$ cercanos a la longitud total de los cromosomas vemos que $d(s)$ prácticamente se detiene debido a que la cromatina, además, está confinada. Sin embargo, es un resultado conocido de geometría que la distancia promedio entre puntos aleatorios dentro de una esfera de radio $R$ es $(36/35)R$, pero $d(s)$ se detiene aproximadamente en los $0.6$ $\mu$m que es bastante inferior al radio del núcleo ($\sim1$ $\mu$m). Esto, de nuevo, es una demostración más de los territorios cromáticos, que limitan el volumen disponible a cada cromosoma a una doceava parte del volumen del núcleo, que equivale a una esfera de $\sim0.5$ $\mu$m de radio.

En la figura \ref{fig:cp_c} en cambio distinguimos, de forma bastante clara, solo dos régimenes. En el primero el exponente con el que decae $P(s)$ es $\sim-2.4$, algo inferior al predicho por la teoría para un modelo FJC con volumen excluido, de nuevo debido a la longitud de persistencia de las partículas LADh. En el segundo, que representa un rango muy grande de valores de $s$, observamos que el exponente es $\sim-0.9$ (que también es compatible con los resultados experimentales \cite{Wang2016}), muy similar al que esperaríamos en un ``glóbulo colapsado'', por lo que concluimos otra vez que este es, de los modelos teórico que hemos visto, el que mejor describe la cromatina.

Como anticipábamos en ambos observables (pero especialmente en $P(s)$) se ve que el cromosoma X, debido a que en proporción tiene muchas más partículas LNDe, se comporta de forma distinta a los demás y adopta configuraciones más extendidas por lo que $d(s)$ es mayor y $P(s)$ es menor en este cromosoma.

\begin{figure}[t]
    \centering
    \includegraphics{../Plots/Simulations/30386-0.200-0.400-0.000-0.000/cm.pdf}
    \caption{Mapa de contactos en la configuración convencional.}
    \label{fig:cm_c}
\end{figure}

Finalmente en la figura \ref{fig:cm_c} mostramos el mapa de contactos que también nos aporta información muy interesante. Vemos primero de todo la clarísima separación de los distintos territorios cromáticos que, a pesar de la longitud de la simulación, sigue siendo muy marcada. Esto está en concordancia con los experimentos y con diversos estudios teóricos. A pesar de que la configuración de equilibrio no tiene territorios cromáticos, el tiempo de relajación del sistema, es decir el tiempo que se espera que haga falta para que los cromosomas se mezclen es de años \cite{Rosa2008} y por supuesto los ciclos celulares son mucho más cortos. Es decir que esta es una característica que no necesita ningún mecanismo que la mantenga de forma activa sino que sobrevive simplemente gracias a la lentitud de la dinámica de los polímeros de estas magnitudes.

El mapa de contactos también nos muestra de forma cristalina la separación de fases dentro de cada cromosoma mediante un patrón de tablero de ajedréz que revela la compartimentalización de los dos tipos de partículas y que por supuesto también se observa experimentalmente. Una ventaja de haber utilizado la secuencia del C. elegans es que en todos los autosomas de esta especie la heterocromatina se concentra en los extremos (ver figura \ref{fig:sequence}) de forma que en el mapa de contactos de cada cromosoma se genera (simplificando) una cuadrícula $3\times3$ en la que las esquinas, que corresponden a interacciones LADh-LADh, son las más intensas como es lógico, y el cuadrado central, que representa interacciones LNDe-LNDe, es menos intenso que las esquinas pero más que el resto de regiones, que representan interacciones cruzadas. Esto es una consecuencia clara de la separación de fases, que lleva a las partículas LNDe a interactuar consigo mismas con mayor frecuencia a pesar de no que exista una interacción atractiva entre las mismas.

\section{Configuración invertida}

Con la siguiente simulación reproduciremos otro tipo de configuración que también se observa en algunas células\footnote{Las células del C. elegans en concreto no adoptan esta configuración pero consideramos que nuestro modelo es suficientemente general como para describir también la organización de otros genomas.} y analizaremos la función de la lámina (o mejor dicho de las interacciones con la misma) en la organización de la cromatina pues vamos a eliminar los sitios de unión a la lámina sin variar el radio del núcleo.

\begin{figure}[p]
    \centering
    \includegraphics[width=0.5\textwidth]{../Multimedia/Images/Inverted-QS.png}
    \caption{Visualización de una configuración invertida de nuestro modelo de la cromatina.}
    \label{fig:vmd_QS_image_i}
\end{figure}

\begin{figure}[p]
    \centering
    \includegraphics{../Plots/Simulations/30386-0.200-0.000-0.000-0.000/rcd.pdf}
    \caption{Densidad de la cromatina en función de la distancia al centro del núcleo en la configuración invertida.}
    \label{fig:rcd_i}
\end{figure}

Vemos en la figura \ref{fig:vmd_QS_image_i} que de esta forma se obtienen configuraciones que denominaremos invertidas pues las partículas LNDe y LADh se intercambian posiciones. Esta peculiar configuración se observa en las células de los bastones de los animales nocturnos pues la aglomeración de la heterocromatina en un único dominio con una posición central alejada de la lámina reduce la dispersión de la luz que lo atraviesa, mejorando la visión nocturna de estos animales \cite{Camara2023}.

En la figura \ref{fig:rcd_i}, donde representamos $\rho(r)$, también observamos que se han invertido las posiciones típicas de ambos tipos de partículas. Sin embargo, también vemos que hay mucho más ruido y las barras de error son mucho mayores, es decir, que las configuraciones obtenidas varían mucho más con las condiciones iniciales y fluctuan mucho más en el tiempo. Esto significa que la lámina ayuda a estabilizar la organización de la cromatina. Esta estabilidad es probablemente una de las múltiples razones por las que en la gran mayoría de células la cromatina interacciona con la lámina y adopta la configuración convencional. También debemos resaltar que en esta gráfica parecería que las partículas LADh, aunque están claramente separadas de la lámina (a la cual se acercan preferentemente las partículas LNDe), no ocupan el centro exacto del núcleo. Sin embargo, la gran incertidumbre en los valores de $\rho(r)$ cerca del centro nos indica que falta estadística y lo más esperable es que, prolongando las simulaciones de forma que este único dominio de partículas LADh tenga más tiempo de difundirse, vieramos que la densidad de este tipo de partículas desciende monótonamente con $r$.

Estos resultados nos indican que la lámina es indispensable para que la cromatina se organice de la forma convencional. Existen fuerzas entrópicas que favorecen que la heterocromatina se coloque cerca de la pared del núcleo \cite{Finan2010}, pero cuando se tiene en cuenta la atracción que tiene la heterocromatina consigo misma y se utilizan valores realistas de su longitud de persistencia, la proporción de ambos tipos de cromatina y el radio del núcleo, estas no son suficientemente fuertes y como hemos visto se obtienen configuraciones invertidas.

\begin{figure}[t]
    \centering
    \includegraphics{../Plots/Simulations/30386-0.200-0.000-0.000-0.000/cm.pdf}
    \caption{Mapa de contactos en la configuración invertida.}
    \label{fig:cm_i}
\end{figure}

Por otro lado los observables $d(s)$ y $P(s)$ apenas cambian con respecto a la configuración convencional, por lo que no mostraremos las gráficas de estos dos observables en esta ocasión. La única diferencia que se puede apreciar es que el cromosoma X está algo más extendido, es decir, que en comparación $d(s)$ es algo mayor y $P(s)$ es algo menor, probablemente por la sencilla razón de que como veremos a continuación todos los cromosomas tienen algo más de movilidad.

Finalmente, en el mapa de contactos representado en la figura \ref{fig:cm_i} vemos los mismos patrones que antes, consecuencia de la separación de fases, pero además vemos claramente que hay muchos más contactos entre cromosomas distintos: los territorios cromáticos son mucho más difusos. Esto nos demuestra que la lámina ayuda a mantener durante más tiempo los territorios cromáticos, como ya habían observado otros estudios computacionales \cite{Kinney2018}, y este hecho es sin lugar a dudas otra de las razones por las que la configuración convencional es mucho más común.
