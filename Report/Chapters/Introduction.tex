\chapter{Introducción}
\label{cap:introduction}

\dots

\section{Motivación}

En el último siglo se ha tendido a diseccionar los sistemas biológicos en sus partes constituyentes más simples para facilitar su entendimiento, de forma similar a como se ha hecho en física. Esta estrategia reduccionista ha sido muy exitosa: ha impulsado el desarrollo de la biología molecular y su resultado más célebre, el descubrimiento de la estructura del ADN, ha sido fundamental para alcanzar la comprensión actual de los mecanismos de la vida. Pero ahora son cada vez más patentes las limitaciones de esta forma de estudiar los seres vivos y, de nuevo como en la física, se está empezando a poner el foco en la complejidad de estos sistemas para intentar comprender sus propiedades emergentes. Esto quiere decir empezar a estudiar las estructuras que se forman de la unión de esas moléculas más pequeñas cuyo comportamiento individual se ha estudiado con tanto detalle, y un ejemplo paradigmático es la cromatina: el complejo de ADN y proteínas que constituye el material génetico del núcleo celular.

\section{Objetivos}

\dots

\section{La física de polímeros}

\dots
