\chapter{Introducción}
\label{cap:introduction}

\section{Motivación y Objetivos}

En el último siglo se ha tendido a diseccionar los sistemas biológicos en sus partes constituyentes más simples para facilitar su entendimiento, de forma similar a como se ha hecho en física. Esta estrategia reduccionista ha sido muy exitosa: ha impulsado el desarrollo de la biología molecular y sus resultados, como el descubrimiento de la estructura del ADN, han sido fundamentales para alcanzar la comprensión actual de los mecanismos de la vida. Pero ahora son cada vez más patentes las limitaciones de esta forma de estudiar los seres vivos y, de nuevo como en la física, se está empezando a poner el foco en la complejidad de estos sistemas para intentar comprender sus propiedades emergentes. Esto quiere decir empezar a estudiar aquellos sistemas formados por muchos de estos subcomponentes (cuyo comportamiento individual se ha estudiado con tanto detalle) y un ejemplo paradigmático de estos es sin duda la cromatina: el complejo de ADN y proteínas que constituye el material génetico del núcleo celular. Este sistema tiene un interés biológico inmenso pues la organización de la cromatina dentro del núcleo influye en muchos procesos biológicos, como por ejemplo la transcripción del ADN.

El principal objetivo del trabajo es la implementación en un programa de simulación de un modelo polimeríco de la cromatina que nos permita reproducir las distribuciones espaciales de la misma que se observan experimentalmente. Para ello, tras una breve introducción a las bases de la física de polímeros, en el capítulo \ref{cap:chromatin} estudiaremos las propiedades más importantes de la cromatina, haremos una revisión de los modelos que se han utilizado hasta ahora para estudiarla y definiremos detenidamente el que utilizaremos nosotros. En el capítulo \ref{cap:methodology} detallaremos como se han realizado estas simulaciones: tanto los algoritmos que se han empleado para la integración de las ecuaciones o el cálculo de fuerzas como el \textit{hardware} del que nos hemos servido. En el capítulo \ref{cap:results} mostraremos finalmente los resultados de las simulaciones y discutiremos el acuerdo entre las mismas y las observaciones experimentales y por último en el capítulo \ref{cap:conclusions} extraeremos algunas conclusiones y mencionaremos algunas posibilidades de trabajo futuro.

\section{La física de polímeros}

Como veremos en detalle en el siguiente capítulo la cromatina a nivel biofísico es, ante todo, un polímero y por ello es indispensable hacer una breve introducción a la física de estas macromoléculas antes de adentrarnos en su modelización. Tal y como indica su nombre, los polímeros (del griego: $\pi o \lambda \upsilon \varsigma$ [polys] ``mucho'' y $\mu \varepsilon \rho o \varsigma$ [meros] ``parte'') son macromoléculas compuestas por muchas unidades repetidas llamadas monómeros, unidas entre sí mediante enlaces covalentes. A pesar de su sencillez, su elevado número de grados de libertad y las ligaduras entre los mismos hace que el espacio de configuraciones de un polímero sea muy extenso y complejo. En esta sección vamos a presentar algunos de los modelos de polímeros más sencillos que aparecen en la literatura y las propiedades más importantes de las configuraciones que adoptan, para posteriormente compararlas con las que obtendremos con nuestro modelo de la cromatina.

Obligatoriamente tenemos que empezar por los llamados modelos ideales, que son aquellos en los que no se tienen en cuenta las interacciones entre monómeros muy separados a lo largo de la cadena (¡aunque estén cerca en el espacio!) y el primero de ellos es el \textit{freely-jointed chain} (FJC). En este modelo no hay interacción alguna entre los monómeros de la cadena, que están unidos por enlaces $\bar{b}_i=\bar{r}_{i+1}-\bar{r}_i$ de longitud fija $l_0$. Este es el modelo más sencillo que existe (no es más que una caminata aleatoria), pero es el punto de partida del resto de modelos y ayuda a entender los fenómenos básicos de la física de polímeros. Es además de los pocos para los que se pueden encontrar soluciones analíticas, gracias al hecho de que su función de partición es factorizable. Aunque visualmente se puedan analizar de forma cualitativa, para caracterizar este modelo y los siguientes cuantitativamente deberemos introducir algunos observables y el primero de ellos será la distancia de extremo a extremo $d_{ee}$ que en el modelo FJC depende del número de monómeros $N$ como \cite{Theodorakopoulos2019},
\begin{equation}
    \label{eq:FJC_dee}
    \langle d_{ee}^2 \rangle=(N-1)l_0^2.
\end{equation}
Como además este modelo es autosimilar (fractal) se cumplirá esta misma relación entre cualesquiera dos monómeros y en general se tendrá que la distancia promedio entre dos monómeros separados a lo largo de la cadena una distancia $s$ es aproximadamente proporcional a $\sqrt{s}$.

Al hacer simulaciones de este y modelo y  (tanto por razones técnicas como para añadir realismo) es considerar que los enlaces son extensibles, es decir, que los monómeros están unidos por un potencial armónico\footnote{Es decir, que están unidos por muelles. Por esta razón los modelos de este tipo se conocen como \textit{bead-spring models}.},
\begin{equation}
    \label{eq:Harmonic}
    V_e=\sum_{i=1}^{N-1}\frac{1}{2}k_e(l_i-l_0)^2
\end{equation}
donde $l_i=||\bar{b}_i||$.

pues para simular modelos con ligaduras como estas (que fijan la longitud de los enlaces) habría que usar algoritmos de integración específicos que las tuvieran en cuenta. En este trabajo hemos preferido no incluir ligaduras de este tipo, además de para no restringir nuestro programa a una clase concreta de modelos, porque los polímeros reales sí que son extensibles. Por ello consideraremos

El modelo de Kratky-Porod o \textit{worm-like chain} discreto (WLC) es una evolución del anterior útil para describir polímeros semi-flexibles donde existe una correlación entre las direcciones de enlaces consecutivos. Con este fin introduce un potencial que depende de los ángulos entre los enlaces $\theta_i$ y tiende a orientarlos en la misma dirección,
\begin{equation}
    \label{eq:WLC_Potential}
    V_b=\sum_{i=1}^{N-2} k_b (1-\cos(\theta_i))
\end{equation}

MENCIONAR QUE WLC TMB ESCALA COM S**0.5 AUNQUE CON OTR CONSTANTE.

MENCIONAR QUE ENLACES EXTENSIBLES NO AFECTA AL ESCALADO, ESTE DEPENDE DE LA CLASE DE UNIVERSALIDAD Y ES EL MISMO PARA TODOS LOS MODELOS IDEALES.

HABLAR DE PROBABILIDAD DE CONTACTO: EN IDEALES -1.5, EN VOL EXCL -2.2.
