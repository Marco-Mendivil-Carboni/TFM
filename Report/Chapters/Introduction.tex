\chapter{Introducción}
\label{cap:introduction}

\dots

\section{Motivación}

En el último siglo se ha tendido a diseccionar los sistemas biológicos en sus partes constituyentes más simples para facilitar su entendimiento, de forma similar a como se ha hecho en física. Esta estrategia reduccionista ha sido muy exitosa: ha impulsado el desarrollo de la biología molecular y sus resultados, como el descubrimiento de la estructura del ADN, han sido fundamentales para alcanzar la comprensión actual de los mecanismos de la vida. Pero ahora son cada vez más patentes las limitaciones de esta forma de estudiar los seres vivos y, de nuevo como en la física, se está empezando a poner el foco en la complejidad de estos sistemas para intentar comprender sus propiedades emergentes. Esto quiere decir empezar a estudiar aquellos sistemas formados por muchos de estos subcomponentes (cuyo comportamiento individual se ha estudiado con tanto detalle) y un ejemplo paradigmático es la cromatina: el complejo de ADN y proteínas que constituye el material génetico del núcleo celular.

Este sistema tiene un interés biológico inmenso pues la correcta transcripción de los genes depende de la organización de la cromatina dentro del núcleo.

AÑADIR MÁS COSAS BIO.

\section{Objetivos}

El principal objetivo del trabajo será la implementación en un programa de simulación de un modelo de la cromatina que permita obtener las distribuciones espaciales de la misma en distintas condiciones, así como el análisis necesario para la caracterización de las mismas. Además, debido a la magnitud del sistema, que hasta hace poco había impedido estudiar el núcleo celular completo mediante simulaciones numéricas, será fundamental desarrollar este programa para que se ejecute en GPUs.

MÁS?

\section{La física de polímeros}

Como veremos en detalle en el capítulo \ref{cap:chromatin} la cromatina a nivel biofísico es, ante todo, un polímero y por ello es indispensable hacer una breve introducción a la física de estas macromoléculas antes de adentrarnos en su modelización.

COPIAR COSAS DEL TFG
