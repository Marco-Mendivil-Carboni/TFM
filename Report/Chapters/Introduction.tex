\chapter{Introducción}
\label{cap:introduction}

Para comprobar la validez de la ecuación de Young-Laplace,
\begin{equation}
  \Delta P=\frac{2\gamma}{R_c},
\end{equation}
debemos estimar la diferencia de presión y el radio de curvatura. Para lo primero basta con calcular $\bar{f}^s_i \ \forall i$, las fuerzas ejercidas por la superficie sobre cada partícula, que incluyen la fuerza repulsiva de la pared y la fuerza atractiva de los enlaces a la lamina. Entonces a cada lado de la apertura,
\begin{equation}
  P=\left\langle-\frac{1}{S}\sum_i\bar{f}^s_i\cdot\hat{n}\right\rangle
\end{equation}
Hay formas equivalentes de calcular la presión pero son más complicadas.

En cambio no está tan claro como estimar $R_c$. Se podría intentar ajustar a la distribución promedio de las partículas una superficie esférica pero no he encontrado la forma de hacerlo y he optado por otro camino. Sea $N$ el número de partículas totales y $N'$ las que han ``translocado'' (aquellas con $||\bar{r}_i||>R_n$), si suponemos que la densidad es la misma a ambos lados de la abertura entonces,
\begin{equation}
  \frac{N-N'}{V_n^a+V_n^b}=\frac{N'}{V_c}\Rightarrow V_c=\frac{V_n^a+V_n^b}{N/N'-1}
\end{equation}
donde claramente $V_n^a+V_n^b=\tfrac{4}{3}\pi R_n^3$. Es un ejercicio de geometría sencillo hallar $V_c$ en función de $R_c$. Resulta que,
\begin{equation}
  V_c=\frac{\pi}{3}R_c^3\left(2\pm\sqrt{1-(R_o/R_c)^2}\right)\left(1\mp\sqrt{1-(R_o/R_c)^2}\right)^2-V_n^b
\end{equation}
donde
\begin{equation}
  V_n^b=\frac{\pi}{3}R_n^3\left(2+\sqrt{1-(R_o/R_n)^2}\right)\left(1-\sqrt{1-(R_o/R_n)^2}\right)^2
\end{equation}
Ahora solo queda invertir esta relación y obtener $R_c$ en función de $V_c$, que a su vez podemos expresar en función de $N'$. Utilizando WolframAlpha obtenemos que,
\begin{equation}
  R_c=\frac{1}{4\tilde{V}}\left(\Gamma^{1/3}+R_o^4+R_o^8\Gamma^{-1/3}\right)
\end{equation}
donde
\begin{equation}
  \Gamma=R_o^{12}+8R_o^6\tilde{V}^2+4\sqrt{R_o^{18}\tilde{V}^2+5R_o^{12}\tilde{V}^4+8R_o^6\tilde{V}^6+4\tilde{V}^8}+8\tilde{V}^4
\end{equation}
y $\tilde{V}=\frac{3}{\pi}\left(V_c+V_n^b\right)$.

\begin{figure}
  \centering
  \begin{tikzpicture}
    \fill [color1!25] (-2.0,2.0) arc (135:405:{2.0*sqrt(2.0)});
    \fill [color2!25] (2.0,2.0) arc (45:135:{2.0*sqrt(2.0)});
    \fill [color3!25] (2.0,2.0) arc (0:180:2.0) arc (135:45:{2.0*sqrt(2.0)});
    \draw [dashed] (0.0,{-2.0*sqrt(2.0)}) -- (0.0,{4.0+2.0*sqrt(2.0)});
    \draw (-2.0,2.0) arc (135:405:{2.0*sqrt(2.0)});
    \draw [dashed] (2.0,2.0) arc (45:135:{2.0*sqrt(2.0)});
    \draw [dashed,latex-latex] (0.0,0.0) -- (2.0,2.0) node[midway,above left] {$R_n$};
    \draw [dashed] (-2.0,2.0) -- (0.0,2.0);
    \draw [dashed,latex-latex] (0.0,2.0) -- (2.0,2.0) node[midway,above] {$R_o$};
    \draw (2.0,2.0) arc (0:180:2.0);
    \draw [dashed,latex-latex] (0.0,2.0) -- ({sqrt(2.0)},{2.0+sqrt(2.0)}) node[midway,above left] {$R_c$};
    \draw (2.0,2.0) arc (-45:225:{2.0*sqrt(2.0)});
    \draw [dashed,latex-latex] (0.0,4.0) -- (2.0,6.0) node[midway,above left] {$R_b$};
    \node [above left] at (-0.5,1.0) {$V_n^a$};
    \node [above left] at (-0.5,2.0) {$V_n^b$};
    \node [above left] at (-0.5,3.0) {$V_c$};
  \end{tikzpicture}
  \caption{Geometría del problema.}
  \label{fig:geometry}
\end{figure}

\begin{figure}
  \centering
  \includegraphics{../Plots/curvature-radius.pdf}
  \caption{Radio de curvatura.}
  \label{fig:R_c}
\end{figure}

\begin{figure}
  \centering
  \includegraphics{../Plots/performance.pdf}
  \caption{Rendimiento.}
  \label{fig:performance}
\end{figure}
