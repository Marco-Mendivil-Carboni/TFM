\chapter{Introducción}
\label{cap:introduction}

\section{Motivación y Objetivos}

En el último siglo se ha tendido a diseccionar los sistemas biológicos en sus partes constituyentes más simples para facilitar su entendimiento, de forma similar a como se ha hecho en física. Esta estrategia reduccionista ha sido muy exitosa: ha impulsado el desarrollo de la biología molecular y sus resultados, como el descubrimiento de la estructura del ADN, han sido fundamentales para alcanzar la comprensión actual de los mecanismos de la vida. Pero ahora son cada vez más patentes las limitaciones de esta forma de estudiar los seres vivos y, de nuevo como en la física, se está empezando a poner el foco en la complejidad de estos sistemas para intentar comprender sus propiedades emergentes. Esto quiere decir empezar a estudiar aquellos sistemas formados por muchos de estos subcomponentes (cuyo comportamiento individual se ha estudiado con tanto detalle) y un ejemplo paradigmático de estos es sin duda la cromatina: el complejo de ADN y proteínas que constituye el material génetico del núcleo celular. Este sistema tiene un interés biológico inmenso pues la organización de la cromatina dentro del núcleo influye en muchos procesos biológicos, como la transcripción del ADN o la división celular.

Por ello este trabajo tiene como objetivo principal el desarrollo de un modelo de la cromatina que nos permita, mediante simulaciones numéricas, reproducir las distribuciones espaciales de la misma que se observan experimentalmente en distintas condiciones. Para ello empezaremos haciendo una breve introducción a la física de polímeros que nos servirá para presentar algunos conceptos importantes. A continuación, en el capítulo \ref{cap:chromatin} estudiaremos las propiedades más importantes de la cromatina, haremos una revisión de los modelos que se han utilizado hasta ahora para estudiarla y definiremos detenidamente el que utilizaremos nosotros. En el capítulo \ref{cap:methodology} detallaremos cómo se han realizado estas simulaciones: cuales son las ecuaciones de la dinámica adecuadas en este contexto, el \textit{hardware} del que nos hemos servido e incluso el algoritmo que hemos utilizado para la parte más costosa computacionalmente del programa. En el capítulo \ref{cap:results} mostraremos finalmente los resultados de las simulaciones en las distintas condiciones que hemos decidido estudiar, explicaremos cómo se han analizado estos resultados y discutiremos el acuerdo entre estos y las observaciones experimentales. Y, por último, en el capítulo \ref{cap:conclusions} valoraremos si hemos cumplido con estos objetivos y mencionaremos algunas posibilidades de trabajo futuro.

\section{La física de polímeros}

Como veremos en detalle en el siguiente capítulo la cromatina a nivel biofísico es, ante todo, un polímero y por ello es indispensable hacer una breve introducción a la física de estas macromoléculas antes de adentrarnos en su modelización. Tal y como indica su nombre, los polímeros (del griego: $\pi o \lambda \upsilon \varsigma$ [polys] ``mucho'' y $\mu \varepsilon \rho o \varsigma$ [meros] ``parte'') son macromoléculas compuestas por muchas unidades repetidas llamadas monómeros, unidas entre sí mediante enlaces covalentes. Lo primero que nos muestra la teoría de polímeros es que la mayoría de sus propiedades físicas son independientes de lo que sean realmente estos monómeros, por lo que aquí consideraremos que un polímero es sencillamente una colección de $N$ ``partículas'' con posiciones $\{\bar{r}_i\}$, unidas sucesivamente, en un medio a temperatura $T$. A pesar de su sencillez, su elevado número de grados de libertad y las ligaduras entre los mismos hace que el espacio de configuraciones de un polímero sea muy extenso y complejo. En esta sección vamos a presentar algunos de los modelos de polímeros más sencillos, pero fundamentales, que aparecen en la literatura y sus propiedades más importantes, centrándonos especialmente en aquellos observables que utilizaremos después para describir la distribución de la cromatina.

Obligatoriamente tenemos que empezar por los llamados modelos ideales, que son aquellos en los que no se tienen en cuenta las interacciones entre monómeros muy separados a lo largo de la cadena (¡aunque estén cerca en el espacio!) y el primero de ellos es el \textit{freely-jointed chain} (FJC). En este modelo no hay interacción alguna entre los monómeros de la cadena, que están unidos por enlaces $\bar{b}_i=\bar{r}_{i+1}-\bar{r}_i$ de longitud fija $l_0$. Este es el modelo más sencillo que existe (no es más que una caminata aleatoria), pero es el punto de partida del resto de modelos y ayuda a entender los fenómenos básicos de la física de polímeros. Es además de los pocos para los que se pueden encontrar soluciones analíticas, gracias al hecho de que su función de partición es factorizable. Aunque visualmente se puedan analizar de forma cualitativa, para caracterizar los distintos modelos cuantitativamente deberemos introducir algunos observables y el primero de ellos será $d(s)$, la distancia (espacial) promedio entre dos monómeros separados una distancia $s$ a lo largo de la cadena, que en el modelo FJC es \cite{Theodorakopoulos2019}
\begin{equation}
    \label{eq:ideal_sd}
    d(s)\propto s^{1/2}.
\end{equation}
Esta ley potencial (o ley de escala) es uno de los resultados más fundamentales de la física de polímeros y nos muestra que las configuraciones de este modelo son autosimilares e invariantes de escala \footnote{Es decir, son fractales, y su dimensión ``fractal'' es 2.}. Otro observable que nos ayudará a describir las configuraciones de estos modelos será $P(s)$, la probabilidad de que dos monómeros separados una distancia $s$ a lo largo de la cadena estén en contacto. Consideraremos que dos monómeros están en contacto si se encuentran a una distancia menor que un determinado radio de corte $r_c$ que especificaremos más adelante. Aunque evidentemente este observable depende de la elección que se haga de $r_c$ su dependencia con $s$ no lo hace y en el modelo FJC se tiene (de forma aproximada) que \cite{Kloczkowski1999}
\begin{equation}
    \label{eq:ideal_cp}
    P(s)\propto s^{-3/2}.
\end{equation}
De nuevo, este observable sigue una ley de escala y su exponente está muy relacionado con el anterior, como se ve con el siguiente argumento heurístico: si en promedio la cadena se aleja de un punto una distancia $d$ $\propto s^{1/2}$ el volumen $V$ visitado por la misma será $\propto d^3$ y la probabilidad de volver al punto de partido será $\propto 1/V$.

Sin embargo, considerar que los enlaces son completamente rígidos es muy idealizado e introduce ligaduras en el modelo que complican su simulación. Por ello es mucho más conveniente y común considerar que los enlaces son extensibles uniendo los monómeros mediante un potencial suave, normalmente un potencial armónico\footnote{Por esta razón los modelos de este tipo se conocen también como \textit{bead-spring models}.},
\begin{equation}
    \label{eq:Harmonic_Potential}
    V_e=\sum_{i=1}^{N-1}\frac{1}{2}k_e(l_i-l_0)^2
\end{equation}
donde $l_i=||\bar{b}_i||$ y $k_e$ es la constante elástica. Además, las configuraciones de este modelo FJC extensible \cite{Fiasconaro2019} en condiciones normales son análogas a las de un modelo FJC con enlaces rígidos de longitud $\langle l \rangle=l_0(1+2/(1+\beta k_e l_0^2))$ (donde $\beta=1/k_BT$ y $k_B$ es la constante de Boltzmann) por lo que, especialmente si $k_e$ es elevada, supone una modificación muy pequeña.

El siguiente es el modelo de Kratky-Porod o \textit{worm-like chain} discreto (WLC), una evolución del anterior con la que podemos describir polímeros semi-flexibles donde existe una correlación entre las direcciones de enlaces consecutivos. Con este fin introduce un potencial que depende de los ángulos entre los enlaces $\theta_i$ y tiende a orientarlos en la misma dirección,
\begin{equation}
    \label{eq:Bending_Potential}
    V_b=\sum_{i=1}^{N-2} k_b (1-\cos(\theta_i))
\end{equation}
donde $\cos(\theta_i)=(\bar{b}_i\cdot\bar{b}_{i-1})/(l_0^2)$. Este potencial dota al polímero de una nueva longitud característica, la longitud de persistencia
\begin{equation}
    \label{eq:WLC_persistence_length}
    l_P=-\frac{l_0}{\ln(\coth(\beta k_b)-1/(\beta k_b))}\approx\beta k_bl_0,
\end{equation}
que es la escala en la que decaen las correlaciones en las orientaciones de los enlaces \cite{Theodorakopoulos2019},
\begin{equation}
    \label{eq:WLC_direction_correlation}
    \langle (\bar{b}_i/l_0)\cdot(\bar{b}_j/l_0) \rangle=e^{-|i-j|l_0/l_P}.
\end{equation}
Por debajo de esta escala el polímero se comporta como una varilla rígida y por lo tanto $d(s)\propto s^{1}$ y $P(s)\propto s^{-3}$ pero para $s$ suficientemente grande las direcciones de los enlaces vuelven a ser independientes y se recuperan los mismos exponentes que en el modelo FJC. De hecho, cuando los efectos de tamaño finito son despreciables (es decir, en el límite termodinámico $N\to\infty$), todos los modelos ideales obedecen las mismas leyes de escala porque pertenecen a la misma clase de universalidad.

Los modelos que sí que pertenecen a otra clase y tienen leyes de escala con exponentes distintos son los que tienen en cuenta las interacciones entre monómeros muy alejados a lo largo de la cadena. Estas interacciones evidentemente existen en la realidad y son esenciales para describir el comportamiento de muchos polímeros reales, incluyendo por supuesto la cromatina. En particular, siempre existirá al menos una fuerza repulsiva que impida que dos monómeros cualesquiera se superpongan, la llamada interacción de volumen excluido. Para introducirla a nuestro modelo añadiremos, de la multitud de potenciales posibles, un potencial de Wang-Frenkel \cite{Wang2020} solo repulsivo,
\begin{equation}
    \label{eq:Wang-Frenkel}
    V_{WF}=\sum_{i=0}^{N-1}\sum_{j=i+1}^{N-1}\varepsilon\left(\left(\frac{\sigma}{d_{ij}}\right)^{2}-1\right)\left(\left(\frac{2\sigma}{d_{ij}}\right)^{2}-1\right)^{2}
\end{equation}
para $d_{ij}\leq\left(2/\sqrt{3}\right)\sigma$ y $-\varepsilon$ de otro modo, donde los parámetros $\varepsilon$ y $\sigma$ regulan la intensidad de la interacción y el tamaño de las partículas respectivamente. Hemos preferido emplear este potencial en lugar de, por ejemplo, el tradicional potencial de Lennard-Jones porque el primero es más estable numéricamente y los resultados que vamos a ver son independientes de la forma exacta del potencial siempre y cuando este impida que se superpongan los monómeros. Desafortunadamente ninguno de los modelos que incluye esta interacción es resoluble analíticamente, pero se ha logrado calcular los exponentes de sus leyes de escala con bastante precisión empleando el formalismo del grupo de renormalización. En estos modelos tendremos que \cite{desCloizeaux1980}
\begin{equation}
    \label{eq:excluded_volume_sd_cp}
    d(s)\propto s^{\nu}  \quad \text{y} \quad P(s)\propto s^{-\nu(3+\theta)}
\end{equation}
donde $\nu\simeq0.588$ y $\theta\simeq0.71$ luego $\nu(3+\theta)\simeq2.2$. Vemos que los exponentes en este caso son mayores, lo que nos indica que las configuraciones que adoptan los polímeros en estos modelos son más extendidas, tal y como era de esperar. Pero como mencionábamos antes también pueden existir interacciones atractivas entre los monómeros, que de hecho serán muy importantes en la cromatina. Podemos añadir muy fácilmente interacciones de este tipo aumentando el radio de corte del potencial de Wang-Frenkel de modo que tenga la forma de la ecuación \ref{eq:Wang-Frenkel} para $d_{ij}\leq2\sigma$ y sea $0$ de otro modo, lo que justifica porque de aquí en adelante tomaremos $r_c=2\sigma$ para la determinación de los contactos entre monómeros. En este caso obtendremos (si $\varepsilon$ es suficientemente grande en comparación con $k_BT$) configuraciones que se llaman de ``glóbulo colapsado'' en las que el polímero ocupa el mínimo espacio posible y por lo tanto
\begin{equation}
    \label{eq:collapsed_globule_sd_cp}
    d(s)\propto s^{1/3}  \quad \text{y} \quad P(s)\propto s^{-1}.
\end{equation}
