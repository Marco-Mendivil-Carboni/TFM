\chapter{Conclusiones y trabajo futuro}
\label{cap:conclusions}

Hemos cumplido con el objetivo principal que nos marcamos al principio del trabajo\footnote{Aunque probablemente no hemos aprovechado todas las posibilidades que nos ofrecía este trabajo.}: el desarrollo de un modelo de la cromatina que nos permitiera, mediante simulaciones numéricas, reproducir las distribuciones espaciales de la misma que se observan experimentalmente. Para ello hemos desarrollado un programa de simulación en el lenguaje CUDA (que se puede consultar en \href{https://github.com/Marco-Mendivil-Carboni/TFM}{\texttt{https://github.com/Marco-Mendivil-Carboni/TFM}}), lo que nos ha permitido simular el número de partículas necesario para modelar adecuadamente este sistema. El modelo que hemos desarrollado es minimalista y genérico e incluye las interacciones básicas para una descripción mesoscópica de la estructura de la cromatina en los organismos eucariotas. Gracias a esa versatilidad hemos podido aplicarlo a la secuencia del C. elegans que es un organismo modelo empleado comúnmente. Nuestras simulaciones han logrado reproducir las características más significativas de la organización de la cromatina, como el comportamiento de la distancia entre monómeros $d(s)$ y la probilidad de contacto $P(s)$ a lo largo de la cadena propios de un ``glóbulo colapsado''. Además, gracias a los mapas de contactos $P(i,j)$ análogos a los experimentos de Hi-C hemos podido observar la fuerte segregación en territorios cromáticos y la separación de fases entre las regiones activas del genoma y las reprimidas. Es más, modificando algunos parámetros del modelo también hemos sido capaces de reproducir configuraciones distintas de la convencional en la que la heterocromatina se concentra cerca de la lámina.

\begin{figure}[p]
    \centering
    \includegraphics[width=1.0\textwidth]{../Multimedia/Images/decay-length.png}
    \caption{Densidad de cromatina (total) normalizada (dividida por su valor máximo) en función de la distancia al centro del núcleo para tres presiones distintas y resultados de la longitud de decaimiento (el parámetro del ajuste a una función exponencial) en los tres casos.}
    \label{fig:exp-fit}
\end{figure}

\begin{figure}[p]
    \centering
    \includegraphics[width=1.0\textwidth]{../Multimedia/Images/bleb.png}
    \caption{Visualización de un corte ecuatorial de la cromatina en un núcleo con un \textit{bleb}: configuración inicial y configuración final.}
    \label{fig:bleb}
\end{figure}

Como consecuencia de este trabajo hemos emprendido algunas líneas de investigación en colaboración con grupos experimentales que están todavía en fase de desarrollo y por ello no hemos incluido en el capítulo anterior de resultados. Sin embargo creemos que es necesario mostrar algunos de estos resultados ahora para resaltar todo lo que se puede investigar mediante modelos como este.

Los experimentos de nanoindentación mediante pinzas ópticas sobre la superficie de un núcleo permiten obtener información sobre las propiedades mecánicas de la cromatina en las cercanías de la lámina. Estos experimentos han sido realizados por el grupo del profesor Horacio López Menéndez que nos ha facilitado algunos resultados experimentales no publicados. Estos muestran que la organización de la cromatina depende del radio del núcleo que a su vez depende de la presión osmótica. De esta forma analizan el decaimiento de la densidad de la cromatina en la cercanías de la lámina y encuentran un comportamiento no monótono del decaimiento de la densidad de cromatina cerca de la lámina dependiendo de las condiciones de estrés osmótico (hipotónico, isotónico e hipertónico).

La mejor forma de reproducir estas condiciones de manera computacional no es evidente y está fuera del objetivo de este trabajo, sin embargo hemos realizado algunas simulaciones intentando implementar estas condiciones de forma sencilla y hemos obtenido los resultados de la figura \ref{fig:exp-fit} que sugieren una dependencia no monotónica del decaimiento de la densidad de cromatina cerca de la lámina.

Otra línea de investigación en colaboración con grupos experimentales consiste en el estudio de la organización de la cromatina en confinamientos más complejos. En concreto el núcleo no siempre es exactamente esférico y a veces tiene deformaciones en forma de vesícula conocidas como \textit{blebs}, que se asocian con varias patologías. Hemos adaptado nuestro programa para poder simular estas geometrías más exóticas del núcleo como se muestra en la figura \ref{fig:bleb}, con el objetivo de analizar como se modifica la organización de la cromatina cuando existe este nuevo espacio disponible dentro del núcleo.

Para finalizar mencionaremos algunas posibilidades más de trabajo futuro que consideramos interesantes y que, por diversos factores, no hemos podido desarrollar en este trabajo. Para empezar, una de las ventajas de hacer simulaciones de dinámica molecular en lugar que de otro tipo es que permite investigar las propiedades dinámicas del sistema, cosa que no hemos hecho en este trabajo y que nos habría permitido ver, por ejemplo, que el movimiento de la cromatina es sub-difusivo \cite{Shi2018}. Por otro lado, aunque nuestro modelo es relativamente sofisticado, no tiene en cuenta ciertos mecanismos como la \textit{loop extrusion} y el \textit{supercoiling} que, son algo más secundarios, pero muchos modelos de la cromatina sí incluyen pues ayudan a describir, entre otras cosas, la formación de los TADs. Por último, hemos considerado por simplicidad que el núcleo era completamente rígido, pero por supuesto tiene cierta elasticidad y por lo tanto también se puede simular su dinámica de forma similar a la de la cromatina para investigar sus propiedades mecánicas \cite{Attar2024}.
