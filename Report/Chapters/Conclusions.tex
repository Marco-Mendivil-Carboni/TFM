\chapter{Conclusiones y trabajo futuro}
\label{cap:conclusions}

\begin{figure}[p]
    \centering
    \includegraphics{../Multimedia/Images/plot-exp-fit.pdf}
    \caption{Densidad de cromatina (total) normalizada (dividida por su valor máximo) en función de la distancia al centro del núcleo para tres presiones distintas.}
    \label{fig:exp-fit}
\end{figure}

Si bien hemos cumplido con el objetivo principal que nos marcamos al principio del trabajo (el desarrollo de un modelo de la cromatina que nos permitiera, mediante simulaciones numéricas, reproducir las distribuciones espaciales de la misma que se observan experimentalmente), probablemente no hemos aprovechado todas las posibilidades que nos ofrecía este trabajo. En especial nos ha faltado estudiar las consecuencias de cambiar el radio del núcleo e investigar la importancia del confinamiento, lo que se ha debido a que no hemos logrado obtener nueva información sobre cómo se organiza la cromatina a partir de las simulaciones que hemos realizado con este objetivo. Aunque sí que hemos obtenido algunos resultados preliminares que que se muestran en la figura \ref{fig:exp-fit} y que sugieren que la rápidez con la que decae la densidad de cromatina depende de la presión y por lo tanto del radio del núcleo.

Aún así, las simulaciones que hemos mostrado sí que nos han servido para comprobar que el modelo que hemos desarrollado contenía los elementos fundamentales que determinan la organización de la cromatina. Hemos logrado reproducir sus características más significativas, como el comportamiento de $d(s)$ y $P(s)$ propios de un ``glóbulo colapsado'', la fuerte separación de los territorios cromáticos y la separación de fases entre las regiones activas del genoma y las reprimidas (que además en la configuración convencional se colocan cerca de la lámina).

\begin{figure}[p]
    \centering
    \includegraphics[width=0.5\textwidth]{../Multimedia/Images/bleb.png}
    \caption{Visualización de la cromatina en un núcleo con un \textit{bleb}.}
    \label{fig:bleb}
\end{figure}

Para finalizar mencionaremos algunas de las posibilidades de trabajo futuro que consideramos más interesantes y que, por diversos factores, no hemos podido desarrollar en este trabajo. Para empezar, una de las ventajas de hacer simulaciones de dinámica molecular en lugar que de otro tipo es que permite investigar las propiedades dinámicas del sistema, cosa que no hemos hecho en este trabajo y que nos habría permitido ver, por ejemplo, que el movimiento de la cromatina es sub-difusivo \cite{Shi2018}. Por otro lado, aunque nuestro modelo es relativamente sofisticado, no tiene en cuenta ciertos mecanismos como la \textit{loop extrusion} y el \textit{supercoiling} que, quizás son algo más secundarios, pero muchos modelos de la cromatina sí incluyen pues ayudan a describir, entre otras cosas, la formación de los TADs. Además, aunque hemos calculado y analizado algunos de los observables más habituales en este campo, en la literatura se han utilizado muchos otros, como por ejemplo la presión ejercida por las paredes del núcleo, que nos habría ofrecido otro tipo de información. Por último, hemos considerado por simplicidad que el núcleo era completamente rígido, pero por supuesto tiene cierta elasticidad y por lo tanto también se puede modelar la lámina de forma similar a la cromatina e investigar sus propiedades mecánicas \cite{Attar2024}.

En relación a esto último también hemos supuesto que la geometría del núcleo era la más sencilla posible, pero este no siempre es exactamente esférico y a veces incluso tiene deformaciones en forma de vesícula conocidas como \textit{blebs}, que se asocian con varias patologías y afectan considerablemnte a la organización de la cromatina. El programa que hemos desarrollado permite la inclusión de un bleb en el núcleo como se ve en la figura \ref{fig:bleb}, lo que lo convierte en la herramienta perfecta para estudiar sus efectos en trabajos futuros.
