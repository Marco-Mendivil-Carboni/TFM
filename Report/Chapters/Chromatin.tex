\chapter{La cromatina}
\label{cap:chromatin}

\section{Propiedades biofísicas}

Como ya mencionamos en la introducción la cromatina es el complejo de ADN y proteínas que se encuentra en el núcleo de las células eucariotas. Esta unión se forma con el principal objetivo de organizar las largas moléculas de ADN en estructuras más densas, lo que previene que se enreden las distintas hebras y que se dañe el ADN además de regular tanto la expresión génica como la replicación del ADN y la división celular.

Introducción a la Cromatina

La cromatina es un complejo de ADN y proteínas que se encuentra en el núcleo de las células eucariotas. Su principal función es empaquetar el material genético de forma compacta para que pueda caber dentro del núcleo celular, mientras se mantiene accesible para los procesos de replicación del ADN, reparación y regulación de la expresión génica. Las proteínas clave en la estructura de la cromatina son las histonas, alrededor de las cuales el ADN se enrolla, formando una estructura denominada nucleosoma.

Histonas y su Función

Las histonas son proteínas básicas que desempeñan un papel crucial en la organización y la regulación de la cromatina. Existen cinco tipos principales de histonas: H1, H2A, H2B, H3 y H4. Las histonas H2A, H2B, H3 y H4 forman el núcleo del nucleosoma, mientras que la histona H1 se asocia con el ADN que conecta los nucleosomas, ayudando a estabilizar la estructura de la cromatina.

1. **Histonas H2A, H2B, H3 y H4:** Estas histonas forman el octámero de histonas alrededor del cual se enrolla aproximadamente 147 pares de bases de ADN, formando el nucleosoma. Cada nucleosoma actúa como una unidad básica de la cromatina, y la disposición de estos nucleosomas determina la compactación del ADN.

2. **Histona H1:** Se conoce como histona de enlace y no forma parte del nucleosoma básico. En cambio, se une al ADN en la entrada y salida del nucleosoma, ayudando a compactar aún más la fibra de cromatina en estructuras superiores, como las fibras de 30 nm.

Las histonas son objeto de diversas modificaciones postraduccionales, como la metilación, acetilación y fosforilación, que influyen en la estructura de la cromatina y regulan la expresión génica al modificar la accesibilidad del ADN.

Tipos de Cromatina

La cromatina se divide en dos tipos principales según su nivel de compactación y su actividad transcripcional: **eucromatina** y **heterocromatina**.

1. **Eucromatina:**
- Es la forma menos compacta de la cromatina y está generalmente asociada con regiones del genoma que son activamente transcritas.
- Es accesible a los factores de transcripción y otras proteínas que regulan la expresión génica, lo que permite la transcripción activa del ADN a ARN.
- Visualmente, la eucromatina aparece como regiones más claras y descompactadas bajo el microscopio electrónico.

2. **Heterocromatina:**
- Es una forma más compacta de la cromatina y se asocia con regiones del genoma que son transcripcionalmente menos activas.
- Se subdivide en **heterocromatina constitutiva** y **heterocromatina facultativa**:
- **Heterocromatina constitutiva:** Está presente en todas las células y contiene secuencias de ADN repetitivo, como los centrómeros y telómeros. Es siempre compacta y generalmente no se transcribe.
- **Heterocromatina facultativa:** Es variable y puede adoptar una forma compacta o extendida dependiendo de la necesidad celular y el contexto del desarrollo.
- Bajo el microscopio electrónico, la heterocromatina aparece como regiones más densamente empaquetadas y oscuras.

Regiones de Interés en la Estructura de la Cromatina

La organización de la cromatina no es aleatoria; existen regiones específicas que juegan roles críticos en la regulación de la estructura del genoma y la expresión génica. Dos de estas regiones de interés son los **LADs** (Dominios Asociados a la Lámina) y los **TADs** (Dominios de Asociación Topológica).

1. **LADs (Lamina-Associated Domains):**
- Los LADs son regiones de la cromatina que interactúan con la lámina nuclear, una red de proteínas ubicada en la periferia interna del núcleo.
- Estas regiones suelen estar asociadas con heterocromatina y se caracterizan por ser transcripcionalmente inactivas.
- Los LADs juegan un papel en la organización del genoma dentro del núcleo, ayudando a mantener ciertos genes silenciados al ubicarlos cerca de la periferia nuclear.

2. **TADs (Topologically Associating Domains):**
- Los TADs son regiones del genoma que se asocian preferentemente consigo mismas, formando dominios estructurales dentro de la cromatina que se mantienen relativamente independientes de otras regiones.
- Dentro de un TAD, los genes y sus elementos reguladores, como los potenciadores, interactúan más frecuentemente entre sí, lo que facilita la regulación específica de la expresión génica.
- Los límites de los TADs son generalmente definidos por sitios de unión de proteínas CTCF y complejos cohesinas, que actúan como barreras para evitar interacciones inapropiadas entre genes y reguladores de diferentes TADs.

Importancia de la Organización de la Cromatina

La organización de la cromatina en eucromatina y heterocromatina, así como la presencia de estructuras como los LADs y TADs, es fundamental para la regulación de la expresión génica y la estabilidad del genoma. La compactación de la cromatina en heterocromatina permite el silenciamiento de genes que no deben ser expresados en ciertos contextos celulares, mientras que la eucromatina permite la activación y acceso a los genes necesarios para las funciones celulares específicas.

Los LADs contribuyen a la segregación espacial del genoma, ayudando a mantener las regiones inactivas alejadas de las activas, mientras que los TADs facilitan interacciones reguladoras dentro de dominios específicos, lo que asegura una expresión génica correcta y controlada.

Conclusión

La cromatina es una estructura compleja y dinámica que juega un papel esencial en la regulación de la expresión génica y la organización del genoma. Las histonas y sus modificaciones postraduccionales permiten la regulación de la compactación de la cromatina, mientras que la división en eucromatina y heterocromatina determina qué regiones del genoma están activas o inactivas. Las regiones específicas como los LADs y TADs reflejan un nivel adicional de organización, asegurando la correcta expresión y funcionamiento del material genético en cada célula.

\begin{figure}
    \centering
    \includegraphics[width=1.0\textwidth]{../Multimedia/Images/Chromatin-Structure.png}
    \caption{\dots \cite{Misteli2020}}
    \label{fig:chromatin-structure}
\end{figure}

\section{Modelo polimérico}

Queda claro por tanto que la cromatina es, en una primera aproximación, un conjunto de polímeros (uno por cromosoma) de gran tamaño confinados dentro de un núcleo aproximadamente esférico. Por ello, esta será la base de nuestro modelo polimérico de la cromatina. Pero en la sección anterior hemos visto que la cromatina no es uniforme, sino que diversas modificaciones químicas a lo largo de la cadena forman dominios mediante mecanismos que todavía no están completamente esclarecidos.

CAMBIAR: Por ello no podremos considerar simples homopolímeros sino que deberemos consider un modelo de copolímeros en los que distintos monómeros tengan distintas propiedades locales. Se han ideado muchos modelos distintos que reflejan distintos aspectos de la cromatina, en este trabajo pretendemos combinar los más importantes con el objetivo de reproducir las características más llamativas de la organización espacial de la cromatina.

Según \cite{Robinson2006} hay 11 nucleosomas por cada 11 nm de fibra de heterocromatina, que tiene 33 nm de ancho.

Consideraremos que cada una de nuestras partículas tiene 33 nucleosomas y, por tanto, un diámetro de 33 nm.

Aunque en el estado eucromático los nucleosomas no están organizados de forma bien definida y estable, interactúan fuertemente de varias maneras formando los llamados TADs (Topologically Associating Domains). Dentro de estos dominios los nucleosomas están presumiblemente muy juntos por lo que suponer que ocupan un volumen similar al del estado heterocromático está justificado.

Como \cite{Bajpai2021} mencionan, podemos más o menos asociar las LADs a la heterocromatina y las no LADs a la eucromatina por lo que en nuestro modelo usaremos sólo dos tipos de partículas por simplicidad:

LADh (A): LAD heterocromatina

LNDe (B): eucromatina no LAD

A pesar de lo que se pueda pensar en un principio según \cite{Bajpai2021} la secuencia exacta de partículas utilizadas en este tipo de modelos no cambia significativamente los resultados, siempre y cuando las propiedades estadísticas de las secuencias sean similares. No obstante, basaremos la secuencia de nuestro modelo de grano grueso en datos experimentales. Utilizaremos los dominios asociados a láminas en el genoma de C. elegans de los que informan \cite{Ho2014} para determinar las partículas LADh de nuestro polímero. Hay varias buenas razones para usar la secuencia de este organismo en particular: es un organismo modelo, su genoma es especialmente corto y sus cromosomas no tienen heterocromatina constitutiva pericentromérica, lo que favorece mucho el uso de modelos como el de \cite{Falk2019} con tres tipos de cuentas.

\begin{figure}
    \centering
    \includegraphics{../Plots/sequence.pdf}
    \caption{\dots}
    \label{fig:sequence}
\end{figure}

También separaremos el polímero al final de cada uno de los 6 cromosomas, cuyas longitudes hemos obtenido de la base de datos GEO NCBI ¿CITAR?.

Dado que las células de C. elegans (como la mayoría de los eucariotas) son diploides dentro del núcleo habrá dos copias de cada cromosoma, lo que hace un total de 12 cromosomas.

Según \cite{Bystricky2004} la heterocromatina tiene una longitud de persistencia de $\sim$ 200 nm lo que en nuestro modelo se traduce en $k_b=6$. Sin embargo, como la heterocromatina sólo representa $\sim$ el 8\% de toda la secuencia y pertenece en la mayoría de los casos a algunos de los dominios LAD (que representan casi la mitad de la secuencia) promediaremos la longitud de persistencia dentro de los dominios LAD y haremos que la partícula LADh tenga $k_b=1$.

Mientras que la euchromatina prácticamente no tiene longitud de persistencia a esta escala por lo que las partículas LNDe tendrán $k_b=0$.

Como explica \cite{Camara2023} la compactación y rigidez de la heterocromatina la hace autoatractiva debido a la atracción por agotamiento. Mientras que la eucromatina es mucho más blanda y no interactúa consigo misma a través de este mecanismo. Cámara también menciona que las regiones inactivas (es decir, los dominios similares a la heterocromatina) tienen una mayor autoafinidad. Así, consideraremos que las partículas LNDe son puramente repulsivas mientras que las partículas LADh son auto-atractivas.

Gracias al maravilloso trabajo de \cite{Falk2019} podemos hacer una estimación de las fuerzas de interacción entre nuestros dos tipos de partículas. En su modelo más elaborado con tres tipos de partículas (una para la eucromatina y dos para la heterocromatina), descubrieron que la eucromatina interactuaba muy poco consigo misma y con los otros tipos de partículas, por lo que podemos considerarla puramente repulsiva por simplicidad. La fuerza de interacción entre las partículas de heterocromatina era diferente para los dos tipos de partículas de heterocromatina, pero su media era del orden de $0.5k_BT$, que es el valor que utilizaremos.

La cromatina tiene otras regiones distinguibles como los centrómeros, los telómeros y las NORs (nucleolus organizer regions) que no consideraremos ya que constituyen fracciones más pequeñas del genoma que la eucromatina y la heterocromatina generales. Por ejemplo, según BioNumbers el nucleolo tiene un radio medio de 0.9 $\text{um}$ y el núcleo de 2.7 $\text{um}$ por lo que las NOR ocupan menos del 4\% del volumen del núcleo a pesar de su función biológica crucial.

El valor exacto de la constante elástica no es realmente relevante en este contexto siempre que sea lo suficientemente alto como para evitar que la distancia entre las partículas enlazadas fluctúe demasiado.

Como hemos visto las propiedades estadísticas de una cadena elástica libremente unida sólo dependen de la constante elástica a través de la fracción por lo que tomar $k_e=128e_u/l_u^2$ debería ser suficiente.

De forma similar pero más simple a \cite{Bajpai2021} añadiremos pozos potenciales a lo largo de la lámina (en posiciones aleatorias) que llamaremos sitios de unión de la lámina. Estos sitios representan los lugares de la lámina a los que un LAD (una partícula LADh en nuestro modelo) puede unirse a través de las llamadas proteínas de unión a la lámina. Haremos estos pozos tan anchos como nuestras partículas y con una profundidad de $8k_BT$ ya que según \cite{Maji2020} utilizando un modelo muy similar las fuerzas de interacción lámina-cromatina que mejor reproducían las observaciones experimentales eran superiores a $2k_BT$ y \cite{Bajpai2021} utilizaron un potencial profundo de $10k_BT$.

Con todo esto en mente tenemos un modelo muy idealizado pero relativamente complejo de la fibra de cromatina confinada dentro de un núcleo esférico. Modulando los parámetros desconocidos de este modelo queremos investigar las diferentes conformaciones que puede adoptar la cromatina y esperamos reproducir las características observadas en los experimentos.

Los parámetros modificables del modelo serán:

El radio del núcleo y la geometría del bleb ($R_n$, $R_o$, $R_b$).

El número de sitios de unión de la lámina ($N'$).

Para nuestra primera simulación daremos a estos parámetros unos valores razonables en condiciones normales. \cite{Ikegami2010} afirman en su artículo que los núcleos de C. elegans tienen un radio de aproximadamente 1 $\text{um}$.
