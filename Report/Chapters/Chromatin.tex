\chapter{La cromatina}
\label{cap:chromatin}

\section{Propiedades y funciones biológicas}

Como ya mencionamos en la introducción la cromatina es el complejo de ADN y proteínas que se encuentra en el núcleo de las células eucariotas. Esta unión se forma con el principal objetivo de organizar las largas moléculas de ADN en estructuras más densas, lo que previene que se enreden las distintas hebras y que se dañe el ADN además de regular tanto la expresión génica como la replicación del ADN y la división celular.

HABLAR DE EUCHROMATINA Y HETEROCHROMATINA Y DE LADS Y DE LA ESTRUCTURA DE LOS CROMOSOMAS.

\section{Modelo polimérico de la cromatina}

Queda claro por tanto que la cromatina es, en una primera aproximación, un conjunto de polímeros (uno por cromosoma) de gran tamaño confinados dentro de un núcleo aproximadamente esférico. Por ello, esta será la base de nuestro modelo polimérico de la cromatina. Pero en la sección anterior hemos visto que la cromatina no es uniforme, sino que diversas modificaciones químicas a lo largo de la cadena forman dominios mediante mecanismos que todavía no están completamente esclarecidos.

CAMBIAR: Por ello no podremos considerar simples homopolímeros sino que debermos consider un modelo de copolímeros en los que distintos monómeros tengan distintas propiedades locales. Se han ideado muchos modelos distintos que reflejan distintos aspectos de la cromatina, en este trabajo pretendemos combinar los más importantes con el objetivo de reproducir las características más llamativas de la organización espacial de la cromatina.

\begin{figure}
    \centering
    \includegraphics{../Plots/sequence.pdf}
    \caption{\dots}
    \label{fig:sequence}
\end{figure}
