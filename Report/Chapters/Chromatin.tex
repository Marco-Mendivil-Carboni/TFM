\chapter{La cromatina}
\label{cap:chromatin}

\section{Propiedades biofísicas}

Como ya mencionamos en la introducción la cromatina es el complejo de ADN y proteínas que se encuentra en el núcleo de las células eucariotas. Esta unión se forma con el principal objetivo de concentrar el material genético para que pueda caber dentro del núcleo celular, pero organizar las largas moléculas de ADN en estructuras más densas también ayuda a prevenir que se enreden las distintas hebras, a evitar que se dañe el ADN y a regular tanto la expresión génica como la replicación del ADN.

\begin{figure}[t]
    \centering
    \includegraphics[width=0.5\textwidth]{../Multimedia/Images/Chromatin-Structure.png}
    \caption{Diagrama esquemático de la estructura de la cromatina a distintas escalas \cite{Felsenfeld2003}.}
    \label{fig:chromatin-structure}
\end{figure}

Las proteínas principales de la cromatina son las histonas, de las que existen cinco clases principales: H1, H2A, H2B, H3 y H4. Dos copias de las histonas H2A, H2B, H3 y H4 forman un octámero alrededor del cual se enrolla el ADN formando la subunidad básica de la cromatina: el nucleosoma. En cada nucleosoma se enrollan $\sim150$ pares de bases y entre un nucleosoma y otro hay $\sim50$ pares de bases de ADN \textit{linker} (al cual se asocia la histona H1 que ayuda a estabilizar la estructura) por lo que se suele considerar que hay un nucleosoma cada $\sim200$ pares de bases.

Además, en ciertas ocasiones estos nucleosomas se ordenan de forma helicoidal como se aprecia en la figura \ref{fig:chromatin-structure} formando una estructura superior conocida como la fibra de 30 nm. Aunque la geometría exacta de esta estructura está a debate hay unos 11 nucleosomas por cada 11 nm de la fibra, que tiene más exactamente 33 nm de ancho \cite{Robinson2006}. Pero no toda la cromatina se encuentra en este estado tan compacto, la gran mayoría no tiene una estructura rígida por encima de la escala del nucleosoma.

Este diferente nivel de compactación nos permite distinguir dos tipos de cromatina\footnote{Históricamente esta fue la primera clasificación que se hizo de las distintas regiones de la cromatina.}: la heterocromatina, más compacta, y la eucromatina, menos compacta. Visualmente, bajo el microscopio electrónico (cuando se utiliza una tinción como la hematoxilina-eosina), las regiones de eucromatina son las más claras y las de heterocromatina, que se tiñen más fuertemente, son las más oscuras.

A menudo las histonas sufren además diversas modificaciones postraduccionales (es decir, cambios químicos ocurridos después de su síntesis por los ribosomas) como la metilación, acetilación, fosforilación o ribosilación. Estas modificaciones se relacionan con varias propiedades de la cromatina, como por ejemplo la tri-metilación de la novena lisina de la histona H3 (frecuentemente abreviada H3K9me3) que se asocia con frecuencia con la heterocromatina.

Este distinto nivel de compactación también se relaciona con una distinta actividad transcripcional. La eucromatina, que es menos compacta, es más accesible a los factores de transcripción y a otras proteínas que regulan la expresión génica, lo que facilita la transcripción del ADN a ARN, mientras que ocurre todo lo contrario en la heterocromatina, que se asocia con regiones del genoma transcripcionalmente menos activas.

También se pueden distinguir dos tipos de heterocromatina: la constitutiva y la facultativa. La primera está formada por las regiones de heterocromatina comunes a todas las células de una especie dada y que habitualmente se concentran en los centrómeros (las constricciones primarias de los cromosomas en la fase mitótica) y los telómeros (los extremos de los cromosomas). La heterocromatina constitutiva por tanto es siempre compacta y generalmente no se transcribe, mientras que el resto de la heterocromatina, la facultativa, es variable y puede adoptar una forma compacta o extendida dependiendo de la necesidad celular y el contexto del desarrollo.

Pero la cromatina se puede clasificar de muchas otras formas. Se pueden identificar regiones de la cromatina que juegan roles específicos en la regulación de la estructura del genoma y la expresión génica y dos de las más importantes son los TADs (Dominios de Asociación Topológica) y los LADs (Dominios Asociados a la Lámina).

Los TADs son regiones del genoma que interaccionan preferentemente consigo mismas, formando dominios estructurales dentro de la cromatina que se mantienen relativamente independientes de otras regiones. Dentro de un TAD, los genes y sus elementos reguladores, como los potenciadores, interactúan más frecuentemente entre sí, lo que facilita la regulación de la expresión génica.

Los LADs son regiones de la cromatina que interactúan (con mayor frecuencia) con la lámina nuclear, una red de proteínas ubicada en la parte interior de la envoltura nuclear en la que está confinada la cromatina. Estas regiones suelen estar asociadas con la heterocromatina y por lo tanto se caracterizan por ser transcripcionalmente inactivas. Los LADs juegan un papel importante en la organización del genoma dentro del núcleo, ayudando a mantener ciertos genes silenciados al ubicarlos cerca de la periferia nuclear y, como veremos más adelante, contribuyendo a la segregación espacial del genoma, es decir, a mantener las regiones inactivas alejadas de las activas.

La cromatina tiene otras regiones distinguibles como los centrómeros, los telómeros y las NORs (Regiones Organizadoras Nucleolares) que no estudiaremos en detalle ni incorporaremos al modelo ya que constituyen fracciones más pequeñas del genoma que las regiones previamente descritas. Por ejemplo, el nucleolo tiene un radio medio de 0.9 $\mu$m y el núcleo de 2.7 $\mu$m \cite{Milo2009} por lo que las NORs ocupan menos del 4\% del volumen del núcleo a pesar de su función biológica crucial.

Antes de pasar a la descripción del modelo polímerico debemos introducir la técnica Hi-C, que será relevante en el capítulo \ref{cap:results}. Este es uno de los métodos experimentales más utilizados para estudiar la organización tridimensional del genoma en el núcleo celular. Esta técnica permite capturar las interacciones físicas entre las diferentes regiones del ADN (es decir, los contactos), revelando cómo se pliega y organiza dentro del espacio nuclear. En Hi-C, el ADN se fija con formaldehído para preservar las interacciones, se fragmenta y se marcan los extremos de los fragmentos cercanos, que luego se ligan para formar moléculas híbridas representativas de las interacciones espaciales. Estas moléculas se secuencian, y los datos obtenidos se analizan para construir mapas de contactos que muestran la frecuencia con la que interaccionan diferentes partes del genoma.

\section{Modelo polimérico}

Queda claro por tanto que la cromatina es, en una primera aproximación, un conjunto de polímeros (uno por cromosoma) de gran tamaño confinados dentro de un núcleo aproximadamente esférico. Por ello, esta será la base de nuestro modelo polimérico de la cromatina. Pero en la sección anterior hemos visto que la cromatina no es uniforme, sino que la propia secuencia del ADN y diversas modificaciones químicas de las histonas generan dominios con distintas propiedades. Por ello no podremos modelar los cromosomas como simples homopolímeros sino que deberemos consider un modelo de copolímeros en los que habrá distintos tipos de monómeros con propiedades locales diferentes.

Por supuesto se han ideado ya muchos modelos distintos que reflejan distintos aspectos de la cromatina. Nosotros vamos a intentar combinar varios de estos modelos buscando un equilibrio entre el realismo del modelo y su simplicidad (que nos ayudará a interpretar los resultados de las simulaciones) con el objetivo de reproducir las características más llamativas de la organización espacial de la cromatina.

\begin{figure}[t]
    \centering
    \includegraphics{../Plots/sequence.pdf}
    \caption{Comparación entre las regiones heterocromáticas y los LADs del C. elegans y la secuencia de nuestro modelo.}
    \label{fig:sequence}
\end{figure}

Para empezar consideraremos que cada una de nuestras partículas contiene 33 nucleosomas (es decir, $\sim6600$ pares de bases) y, por tanto, un diámetro de 33 nm. Aunque en el estado eucromático los nucleosomas no están organizados de forma bien definida y estable, hemos visto que interactúan de varias maneras formando los TADs, dentro de los cuales los nucleosomas están relativamente cerca, por lo que suponer que ocupan un volumen similar al del estado heterocromático está justificado.

Como podemos asociar los LADs con la heterocromatina y los dominios no asociados a la lámina con la eucromatina \cite{Bajpai2021} en nuestro modelo, por simplicidad, combinaremos ambas clasificaciones y usaremos sólo dos tipos de partículas:
\begin{enumerate}
    \item LADh (A): Dominios Asociados a la Lámina heterocromáticos
    \item LNDe (B): Dominios No asociados a la Lámina eucromáticos
\end{enumerate}

A pesar de lo que se pueda pensar en un principio la secuencia exacta de partículas utilizadas en este tipo de modelos no cambia significativamente los resultados \cite{Bajpai2021}, siempre y cuando las propiedades estadísticas de las secuencias sean similares. No obstante, basaremos la secuencia de nuestro modelo en datos experimentales. Utilizaremos los LADs del genoma del C. elegans \cite{Ho2014,C.elegans1998} para determinar las partículas LADh de nuestro modelo. Hay múltiples buenas razones para usar la secuencia de este organismo en particular: es un organismo modelo, su genoma es especialmente corto y sus cromosomas no tienen heterocromatina constitutiva pericentromérica, la cual incentiva el uso de modelos más complejos con tres tipos de partículas \cite{Falk2019}. En la figura \ref{fig:sequence} se aprecia que efectivamente en este genoma hay una correlación entre las regiones heterocromáticas (que se deducen de la presencia de la modificación H3K9me3) y los LADs, que se concentran en los extremos de cada cromosoma (la cual es una característica peculiar de este organismo pero no alterará los resultados como veremos más adelante). También se ve como las partículas de nuestro modelo son suficientemente pequeñas como para reproducir con gran precisión cada uno de los LADs, que suelen tener tamaños mucho mayores de 6 kb. Por otro lado, dado que las células del nematodo C. elegans (como la mayoría de células eucariotas) son diploides, dentro del núcleo colocaremos dos copias de cada uno de estos cromosomas, lo que hace un total de 12 cromosomas.

Por otra parte, la heterocromatina tiene una longitud de persistencia de $\sim200$ nm \cite{Bystricky2004} lo que en nuestro modelo se traduce en $k_b=6$. Sin embargo, como la heterocromatina sólo representa $\sim$ el $8\%$ de toda la secuencia y pertenece en la mayoría de los casos a algún LAD (que representan casi la mitad de la secuencia) promediaremos la longitud de persistencia dentro de los LADs y haremos que las partículas LADh tengan $k_b=1$. Mientras que la eucromatina prácticamente no tiene longitud de persistencia a esta escala por lo que las partículas LNDe tendrán $k_b=0$.

También se observa que las regiones inactivas (es decir, los dominios similares a la heterocromatina) tienen una mayor autoafinidad \cite{Camara2023}, entre otros motivos, porque la mayor compactación y rigidez de la heterocromatina provoca atracción por depleción mientras que la eucromatina es mucho más blanda y no interactúa consigo misma a través de este mecanismo. Afortunadamente podemos hacer una estimación de las fuerzas de interacción entre nuestros dos tipos de partículas. En \cite{Falk2019}, que utilizaba un modelo bastante más complejo con tres tipos de partículas (una para la eucromatina y dos para la heterocromatina), encontraron que la eucromatina interactuaba muy poco consigo misma y con los otros tipos de partículas, por lo que podemos considerar que las partículas LNDe son puramente repulsivas por simplicidad. En el mismo artículo vieron que la fuerza de interacción entre las partículas de heterocromatina era algo diferente para los dos tipos de partículas de heterocromatina pero su promedio era del orden de $0.5k_BT$, que es el valor que utilizaremos para nuestras partículas LADh.

El valor exacto de la constante elástica no es realmente relevante en este contexto siempre que sea lo suficientemente alto como para evitar que la distancia entre las partículas enlazadas fluctúe demasiado y a la vista de los resultados que presentamos en el capítulo \ref{cap:introduction} tomar $k_e=128e_u/l_u^2$ debería ser suficiente.

Para modelar la atracción de los LADs a la lamina, de forma similar pero más simple a \cite{Bajpai2021}, añadiremos pozos de potencial\footnote{Con una forma prácticamente armónica en el centro pero que se conecta de forma suave con el exterior del pozo: $V=\sum_{i=0}^{N-1}\sum_{j=0}^{N'-1}\varepsilon'\left(\frac{4}{3}\frac{{d'}_{ij}^{2}}{r_c^{2}}-\frac{1}{3}\frac{{d'}_{ij}^{8}}{r_c^{8}}-1\right)$.} (que llamaremos sitios de unión a la lámina) en posiciones aleatorias de la misma. Estos sitios representan los lugares de la lámina a los que un LAD (una partícula LADh en nuestro modelo) puede unirse a través de las llamadas proteínas de unión a la lámina. Haremos estos pozos tan anchos como nuestras partículas y con una profundidad de $8k_BT$ ya que según \cite{Maji2020} utilizando un modelo muy similar las fuerzas de interacción lámina-cromatina que mejor reproducían las observaciones experimentales eran superiores a $2k_BT$ y \cite{Bajpai2021} utilizaba un potencial de $10k_BT$ de profundidad.

Con todo esto en mente tenemos un modelo muy idealizado pero relativamente complejo de la fibra de cromatina confinada dentro del núcleo, cuyos parámetros hemos resumido en la tabla \ref{tab:parameters}. Modulando los parámetros desconocidos de este modelo queremos investigar las diferentes conformaciones que puede adoptar la cromatina y esperamos reproducir las características observadas en los experimentos. En concreto los únicos parámetros modificables del modelo serán:
\begin{enumerate}
    \item El radio del núcleo ($R_n$).
    \item El número de sitios de unión a la lámina ($N'$).
\end{enumerate}

\begin{table}
    \centering
    \begin{tabular}{c c c}
        \toprule
        parámetro       & valor (SI)         & valor (UR)          \\
        \midrule
        $l_0$           & $33$ nm            & $1$ $l_u$           \\
        $k_e$           & $0.48$ pN/nm       & $128$ $e_u$/$l_u^2$ \\
        $k_b^A$         & $4.11$ pN$\cdot$nm & $1$ $e_u$           \\
        $\sigma$        & $33$ nm            & $1$ $l_u$           \\
        $\epsilon^{AA}$ & $2.06$ pN$\cdot$nm & $0.5$ $e_u$         \\
        $\epsilon'$     & $32.9$ pN$\cdot$nm & $8$ $e_u$           \\
        \bottomrule
    \end{tabular}
    \caption{Parámetros de nuestro modelo de la cromatina en unidades del sistema internacional (SI) y en unidades reducidas (UR), las unidades del programa.}
    \label{tab:parameters}
\end{table}
