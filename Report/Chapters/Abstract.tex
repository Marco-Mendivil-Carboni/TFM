\section*{Resumen}
\label{cap:abstract}

En el último siglo se ha tendido a diseccionar los sistemas biológicos en sus partes constituyentes más simples para facilitar su entendimiento. Esta estrategia reduccionista ha sido muy exitosa pero ahora son cada vez más patentes las limitaciones de esta forma de estudiar los seres vivos y se está empezando a poner el foco en la complejidad de estos sistemas para intentar comprender sus propiedades emergentes. Esto quiere decir empezar a estudiar aquellos sistemas formados por muchos de estos subcomponentes y un ejemplo paradigmático de estos es sin duda la cromatina: el complejo de ADN y proteínas que constituye el material génetico del núcleo celular. Este sistema tiene un interés biológico inmenso pues la organización de la cromatina dentro del núcleo influye en muchos procesos biológicos, como la transcripción del ADN o la división celular. Por ello este trabajo tiene como objetivo principal el desarrollo de un modelo de la cromatina que nos permita, mediante simulaciones numéricas, reproducir las distribuciones espaciales de la misma que se observan experimentalmente en distintas condiciones.
